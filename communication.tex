Communication

	La communication étant importante pour la réalisation du projet, nous sommes partis sur l'élaboration
d'une affiche A3.

	Nous avons souhaité que cette affiche se devait d’être contrastée et peu chargée afin qu’elle soie
claire et lisible de loin, c’est pourquoi la combinaison du noir sur fond blanc fut adaptée à ce choix.
De plus, il fallait que l’identité visuelle de l’affiche puisse parler aux joueurs, d’où l’utilisation
d’une police spéciale de grande taille, police de la licence « Starcraft », et placée en haut de l’affiche.

	La date et le lieu bénéficient également d’une taille plus grande que le texte du corps de l’affiche.
Concernant ce dernier, nous avons décidé de ne pas le charger en texte afin de bénéficier d’une police
plus grande grâce à l’espace obtenu.
	Nous avons mis en avant la venue du commentateur et joueur Soey  de niveau « top master »  dans le but
de donner une plus grande crédibilité à l’évènement.
	Enfin, pour rajouter à l’identité visuelle, nous avons fait figurer  les emblèmes des trois races du jeu,
ainsi qu’une grande bannière où apparaît le protagoniste principal de la prochaine extension.
21 affiches ont été imprimées au format A3, pour ensuite être affichées dès le 1er mars à l’IUT A, au M1,
au M3, au M5, à la MDE et dans quelques résidences universitaires.

	L’affiche seule n’étant pas suffisante pour la communication, nous nous sommes tournés vers la
Maison Des Étudiants et son mailing afin de diffuser l’information à tous les étudiants de l’université.
Profitant de ce moyen, nous avons élaboré un questionnaire. Le but de celui-ci était d’obtenir une estimation
du nombre de participants et trouver un potentiel commentateur.

	Malheureusement, le mail n’a été envoyé que le 5 mars, soit une semaine avant l’évènement.
Nous avions néanmoins déjà trouvé notre commentateur.

Le questionnaire comporte quatre à neuf questions selon les réponses des sondés. 30 ont répondu.
Voici les réponses à quelques questions :


Êtes-vous intéressé par une soirée Starcraft 2 (WoL, HotS) ?
[graphique]
La soirée se déroulera un mercredi soir (18h-23h), serez-vous présent ?
[graphique]
Aimeriez-vous y jouer au cours de cette soirée ?
[graphique]

	Nous pouvons tirer de ces informations une estimation de plus ou moins 15 personnes présentes si l’on prend
la moitié des « peut-être » + les « oui ».
	En sachant que tous les étudiants ne prêtent pas attention aux mailings de l’université, on peut ajouter
une quinzaine de personnes pour atteindre une estimation de 30 participants.

	Un autre moyen de communication fut utilisé, la création d'un évènement Facebook. Néanmoins cela n'a pas fonctionné,
la propagation de cet évènement ne fut pas suffisante sur le réseau social.


Contraintes :

%TODO
