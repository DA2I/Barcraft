Au final, bien que la soirée n'amena pas le nombre de participants
escompté, nous sommes heureux que le projet ait pu aboutir.  Jusqu'au
dernier moment nous avons tout fait pour que la soirée puisse avoir
lieu, en trouvant des contournements aux obstacles rencontrés. Par
exemple, la réaction face à l'annulation de la venue des commentateurs.

Bien sûr, nous avons fait des erreurs et la soirée aurait pu mieux être
préparée. Nous aurions dû commencer bien plus tôt notre démarche
concernant la demande de financement auprès de la FSDIE, nous avons donc
été obligés d'avoir recours au budget de la formation pour acheter les
plateaux repas. Cela impliquait également de finir l'organisation de la
soirée pour le début de février, la FSDIE étant stricte à propos de la
constitution du dossier de financement. Nous aurions dû peut-être aussi
accepter l'offre de la team \og aAa \fg{} qui consistait à retransmettre
la soirée sur leur chaine télévisuel en ligne. Peut-être que ça aurait
apporté plus de visibilité à l'évenement sur la scène \og eSportive
\fg{}.

Notre principale erreur était de penser que l'on aurait eu le temps de
tout faire à partir de janvier. En commencant plus tôt, nous aurions pu
faire venir des commentateurs/joueurs plus connus, et surtout commencer
la communication plus tôt même si la soirée n'aurait peut-être pas eu
beaucoup plus de succès à cause des intempéries.
