\section{Jeux de stratégie en temps réel}%{{{
\label{sec:jeux_de_strategie_en_temps_reel}

\lettrine{U}{ne} très bonne définition de ce qu'est le jeux de stratégie en
temps réel est consultable sur Wikipédia :

\begin{quote}

Le jeu de stratégie en temps réel (STR, ou RTS pour la
dénomination du genre en anglais : real-time strategy) est un type de
jeu de stratégie particulier qui notamment, et par opposition au jeu de
stratégie au tour par tour, n’utilise pas un découpage arbitraire du
temps [\ldots]

Dans le tour par tour, donc, l'issue d’une confrontation est résolue par
étapes, un combat en succédant un autre, de manière à laisser à chaque
joueur le temps de réfléchir sur la prochaine étape. Une confrontation
est alors résolue par combats successifs et isolés.

Mais le problème qui se pose lorsque l'on désire simuler des affrontements
réalistes est qu’en général, les unités n’attendent pas que ce soit leur
tour pour attaquer et gérer l'affrontement de plusieurs unités en même
temps et de façon réaliste relève du domaine de l'impossible pour l'être
humain, jusqu'à l'invention des processeurs. Ceux-ci, programmés de
manière à simuler un affrontement peuvent désormais gérer le déplacement
de plusieurs unités simultanément et résoudre les combats de façon
simultanée et rigoureuse dans le cadre d’un jeu vidéo.

\end{quote}

% section jeux_de_strategie_en_temps_reel (end)%}}}
\section{Le sport électronique (eSport)}%{{{
\label{sec:le_sport_electronique_esport_}

\lettrine{L}{’eSport} est un terme désignant le
système compétitif mis en place dans le domaine des jeux-vidéo
multi-joueurs. À la manière des sports physiques, l'eSport s’est
professionnalisé à partir de la fin des années 90. Il existe donc des
équipes de joueurs professionnels pouvant posséder un entraineur, un
manager ou encore des sponsors. Par exemple, la somme totale des
récompenses mises en jeu pour Starcraft 2 sur toute l’année 2011 atteint
2,5 millions d’euros.  Mais l'eSport en est encore à ses débuts,
l'absence d'une autorité « eSportive » laisse subsister des délais de
paiement ainsi que des non payés.

% section le_sport_electronique_esport_ (end)%}}}
\section{La licence Starcraft}%{{{
\label{sec:la_licence_starcraft}

\subsection{Starcraft \& Starcraft : Brood War}%{{{
\label{sub:starcraft_&_starcraft_brood_war}

\lettrine{S}{tarcraft} est un jeu vidéo de stratégie en temps réel (STR) développé
par Blizzard Entertainment. Le premier opus est sorti en Europe le 31
mars 1998 sur PC. Il s'inscrit dans la lignée des jeux de stratégie de
Blizzard maintenant très populaire auprès des amateurs du genre. Avec
plus de 11 millions de copies vendues dans le monde, il est un des
jeux vidéo sur PC les mieux vendus et reste à ce jour le jeu de
stratégie en temps réel le plus vendu de tout les temps.

(image)

% subsection starcraft_&_starcraft_brood_war (end)%}}}
\subsection{Starcraft 2 : Wings of Liberty}%{{{
\label{sub:starcraft_2_wings_of_liberty}

\lettrine{C}{elui-ci} sort dans le monde après 12 ans d'attente
interminable le 27 juillet 2010. Blizzard ayant fait le choix marketing
de faire trois campagnes distinctes, il y aura deux extensions de
prevues ce qui à fait polémique à l'époque. Avec le deuxième opus,
blizzard à fait le pari de démocratiser les jeux de stratégie en temps
réel en rendant WoL plus facile d'accès comparé a Brood War qui lui
requiérait beaucoup d'APM\, \footnote{Actions par minute, autrement dit
la faculté d'exécuter des tâches avec le clavier et la souris en un
temps donné.}. Pari reussi puisqu'il connait
rapidement un important succès commercial avec plus de 4,5 millions de
copies vendues 6 mois après la sortie du jeu.

(image)

% subsection starcraft_2_wings_of_liberty (end)%}}}
\subsection{Starcraft 2 : Heart of the Swarm}%{{{
\label{sub:starcraft_2_heart_of_the_swarm}

\lettrine{H}{eart} of the Swarm est la première des deux extensions
prévu du jeux Starcraft 2. Elle constitue la suite du premier chapitre
\emph{Wings of Liberty}. La date de sortie officielle est le 12 mars 2013 soit
deux ans et demi après le premier opus. Aucune statistique de vente n'a
encore été publiée par Blizzard actuellement.

% subsection starcraft_2_heart_of_the_swarm (end)%}}}

% section la_licence_starcraft (end)%}}}
\section{League of Legends}%{{{
\label{sec:league_of_legends}

League of Legends est un jeu de stratégie en équipe de type
MOBA\, \footnote{Multiplayer Online Battle Arena ou Arène de bataille en
ligne multijoueur qui est un mix entre RTS et jeu de rôle.}. Au début d'une partie, quelque
soit le mode de jeu, chaque joueur choisit un héros qu'il va représenter
sur le champ de bataille. Les aspects coordination et travail d'équipe
sont très importants pour le déroulement de la partie. Tout un panel de
stratégies de jeu sont à notre disposition afin de prendre l'avantage sur
l'équipe adverse, le but étant de détruire leur base afin de
gagner la partie.

% section league_of_legends (end)%}}}
\section{Barcraft}%{{{
\label{sec:barcraft}

\lettrine{C}{omme} sous nom l'indique, il s'agit d'organiser une soirée
autour du jeu \emph{Starcraft} tout en bénéficiant de l'ambiance d'un bar. Pour cela, nous
faisons appel à des intervenants extérieurs. Afin d'offrir un maximum
de spectacle, nous avons demandé à des joueurs professionnels ainsi que
des commentateurs de participer à l'événement. La soirée
se déroule sous la forme de matchs commentés. Pour une bonne intéraction
avec le public, les matchs sont vidéoprojetés sur un écran géant. Au
cours de la soirée les joueurs du public pourront prendre place et jouer
contre les joueurs professionnels.

% section barcraft (end)%}}}
