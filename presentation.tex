\subsection{Jeux de stratégie en temps réel}%{{{
\label{sub:jeux_de_strategie_en_temps_reel}

\lettrine{U}{ne} très bonne définition de ce qu'est le jeux de stratégie en
temps réel est consultable sur wikipédia :

  Le jeu de stratégie en temps réel (STR, ou RTS pour la
  dénomination du genre en anglais : real-time strategy) est un type de
  jeu de stratégie particulier qui notamment et par opposition au jeu de
  stratégie au tour par tour n’utilise pas un découpage arbitraire du
  temps [...]

  Dans le tour par tour, donc, l'issue d’une confrontation est résolue par
  étapes, un combat en succédant un autre, de manière à laisser à chaque
  joueur le temps de réfléchir sur la prochaine étape. Une confrontation
  est alors résolue par combats successifs et isolés.

  Mais le problème qui se pose lorsqu'on veut simuler des affrontements
  réalistes est qu’en général, les unités n’attendent pas que ce soit leur
  tour pour attaquer et gérer l'affrontement de plusieurs unités en même
  temps et de façon réaliste relève du domaine de l'impossible pour l'être
  humain, jusqu'à l'invention des processeurs. Ceux-ci, programmés de
  manière à simuler un affrontement peuvent désormais gérer le déplacement
  de plusieurs unités simultanément et résoudre les combats de façon
  simultanée et rigoureuse dans le cadre d’un jeu vidéo.

% subsection jeux_de_strategie_en_temps_reel (end)%}}}
\subsection{Le sport éléctronique (eSport)}%{{{
\label{sub:le_sport_electronique_esport_}

\lettrine{L}{’eSport} (ou sport électronique) est un terme désignant le
système compétitif mis en place dans le domaine des jeux-vidéo
multi-joueurs.  A la manière des sports physiques, l'eSport s’est
professionnalisé à partir de la fin des années 90. Il existe donc des
équipes de joueurs professionnels pouvant posséder un entraineur, un
manager ou encore des sponsors. Par exemple, la somme totale des
récompenses mises en jeu pour Starcraft 2 sur toute l’année 2011 atteint
2,5 millions d’euros.  Mais l'eSport en est encore à ses débuts,
l'absence d'une autorité « eSportive » laisse subsister des délais de
paiement ainsi que des non payés.

% subsection le_sport_electronique_esport_ (end)%}}}
\subsection{La licence Starcraft}%{{{
\label{sub:la_licence_starcraft}

\subsubsection{Starcraft \& Starcraft : Brood War}%{{{
\label{ssub:starcraft_&_starcraft_brood_war}

\lettrine{S}{tarcraft} est un jeu vidéo de stratégie en temps réel (STR) développé
par Blizzard Entertainment. Le premier opus est sorti en Europe le 31
mars 1998 sur PC. Il s'inscrit dans la lignée des jeux de stratégie de
Blizzard maintenant très populaire auprès des amateurs du genre. Avec
plus de 11 millions de copies vendues dans le monde, il est un des
jeux vidéo sur PC les mieux vendus et reste à ce jour le jeu de
stratégie en temps réel le plus vendu de tout les temps.

(image)

% subsubsection starcraft_&_starcraft_brood_war (end)%}}}
\subsubsection{Starcraft 2 : Wings of Liberty}%{{{
\label{ssub:starcraft_2_wings_of_liberty}

\lettrine{C}{elui}-ci sort dans le monde après 12 ans d'attente
interminable le 27 juillet 2010. Blizzard ayant fait le choix marketing
de faire trois campagnes distinctes, il y aura deux extensions de
prevues ce qui à fait polémique à l'époque. Avec le deuxième opus,
blizzard à fait le pari de démocratisé le jeux de stratégie en temps
réel en rendant WoL plus facile d'accès comparées a Brood War qui lui
requiérait beaucoup d'APM (nbp). Pari reussi puisqu'il connait
rapidement un important succès commercial avec plus de 4,5 millions de
copies vendues 6 mois après la sortie du jeu.

(image)

% subsubsection starcraft_2_wings_of_liberty (end)%}}}
\subsubsection{Starcraft 2 : Heart of the Swarm}%{{{
\label{ssub:starcraft_2_heart_of_the_swarm}

\lettrine{H}{eart} of the Swarm est la première des deux extensions
prévu du jeux Starcraft 2. Elle constitue la suite du premier chapitre
Wings of Liberty. La date de sortie officielle est le 12 mars 2013 soit
deux ans et demi après le premier opus.  Aucune statistique de vente n'a
encore été publiée par blizzard actuellement.

% subsubsection starcraft_2_heart_of_the_swarm (end)%}}}

% subsection la_licence_starcraft (end)%}}}
\subsection{League of Legends}%{{{
\label{sub:league_of_legends}

League of Legends est un jeu de stratégie en équipe de type
MOBA(Multiplayer Online Battle Arena). Au début d'une partie, quelque
soit le mode de jeu, chaque joueur choisit un héros qu'il va représenter
sur le champ de bataille. Les aspects coordination et travaille d'équipe
sont très important pour le déroulement de la partie. Tout un panel de
stratégie de jeu sont à notre disposition afin de prendre l'avantage sur
l'équipe adverse, le but étant de détruire la base adverse afin de
gagner la partie.

% subsection league_of_legends (end)%}}}
\subsection{Barcraft}%{{{
\label{sub:barcraft}

\lettrine{C}{omme} sous nom l'indique, il s'agit d'organiser une soirée
autour du jeu Starcraft dans l'ambiance d'un bar.	Pour cela, nous
faisons appelle à des intervenants extérieurs. Afin d'ouvrir un maximum
de spectacle, nous avons demander à des joueurs professionnels ainsi que
des commentateurs professionnels de participer à l'événement.  La soirée
se déroule sous la forme de matchs commentés. Pour une bonne interaction
avec le public, les matchs sont vidéo projetés sur un écran géant. Au
cour de la soirée les joueurs du public pourront prendre place et jouer
contre nos joueurs professionnels.

% subsection barcraft (end)%}}}
