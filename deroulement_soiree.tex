 Deroulement de la soirée 

Previsionnel :

La date de la soirée devait être judicieusement choisie afin obtenir une affluence satisfaisante.

Ayant vu grâce au mailing de l’université que la MDE accueillait des soirées chaque jeudi 
nommées « Jeudifférents », nous nous sommes dit qu’une soirée comme la notre collait 
parfaitement à appellation. Malheureusement, nous avions commencé à prendre contact 
début février pour la réservation, ce qui fut trop tard. Tous les jeudis sur deux mois étaient 
déjà réservés.
Il nous a donc fallu choisir un autre jour de la semaine. Après renseignements auprès de la 
MDE, seuls le mercredi et le vendredi étaient libres la semaine du 11 au 17 mars. Le choix ne 
fut pas difficile, le mercredi fut choisi naturellement car les étudiants ayant un appartement 
dans la métropole rentrent chez eux le vendredi. De plus, le mercredi 13 mars correspondait 
au lendemain de la sortie de la nouvelle extension du jeu.
A propos de l’horaire de la soirée, nous avions décidé de commencer dès 18h pour finir à 
22h. En effet, le fait de commencer à 18h permet aux étudiants de se rendre à la soirée juste 
après les cours. Ensuite, d’après l’AEI, la MDE ferme à 23h et pour ouvrir au-delà de cette 
heure, il était nécessaire d’avoir une dérogation pénible à obtenir. Nous nous sommes donc 
donné une heure afin d’avoir le temps de remballer, mais également de partir plus tôt pour 
reconduire les intervenants chez eux, à Dunkerque.
Après s’être mis d’accord sur la date et l’heure de la soirée, nous devions discuter du 
programme.
Il ne fut pas aisé de définir un programme à heures fixes, car les parties n’ont pas une durée 
définie. Elles peuvent durer entre 5 et 50 minutes, la moyenne se situant à 25 minutes de jeu. 
C’est pourquoi nous avons décidé d’énoncer le contenu de la soirée sans y faire figurer 
d’heure.
Nous avions donc prévu de commencer par 20 minutes d’introduction en vidéo sur le jeu et 
ses mécanismes afin de pouvoir rendre la soirée plus compréhensible pour les néophytes. 
Puis enchainer avec les matchs commentés des intervenants pour finir avec des matchs 
confrontant des joueurs du public. Nous avons choisis cet ordre car nous pensions que le plus 
amusant était de faire jouer le public, et que cela permettrait de les retenir un maximum.
Imprévus majeurs :
Arrivé le jour J, nous avons été confrontés à plusieurs imprévus.
La météo ne fut pas clémente le 12 mars ainsi que le 13 mars, la neige étant fortement 
tombée ces jours là. La circulation était devenue difficile dans la région et beaucoup de cours 
furent perturbés.
Ceci impacta fortement l’affluence de la soirée, et surtout son déroulement. Les intervenants 
ne purent pas venir du fait des conditions climatiques. Le retour à la fin de la soirée aurait été 
dangereux à cause du verglas, et aucun hébergement n’eut pu être possible.
Ces imprévus nous ont donc forcés à improviser.

La soirée :
Les préparatifs commencèrent à 16h, tout se passa bien ou presque. Nous nous aperçûmes 
qu’il fallût utiliser une connectique autre que VGA afin de connecter nos PC au 
vidéoprojecteur de la MDE. Heureusement, le directeur de la MDE M.Bross possédait un 
vidéoprojecteur VGA avec une bonne qualité de projection.
Une dizaine de personnes arrivèrent passé 18h. Comme prévu à cause du climat, peu de 
personnes assistèrent à la soirée. Nous avons recensé une vingtaine de personnes 
différentes, mais jamais plus de 11 personnes simultanément.
A cause des imprévus et compte tenu de la faible affluence, nous avons décidé de faire jouer 
directement les personnes présentes en projetant leurs parties sur vidéoprojecteur comme 
prévu. Julien étant parmi nous le plus expérimenté, il put donc commenter quelques matchs 
au cours de la soirée.
Vers 18h50, nous fûmes encore victime d’un imprévu. Les serveurs de jeu de tous les jeux en 
ligne de Blizzard tombèrent, nous empêchant de jouer pendant 30 minutes. Pendant ce 
temps, nous diffusions les vidéos explicatives de Starcraft II, ce qui permit de combler le 
blanc engendré. Néanmoins, un groupe de 4 personnes s’en alla durant ce laps de temps.
Une fois la situation rentrée dans l’ordre, nous pûmes continuer la soirée en petit comité et 
terminer à 23h. Entre temps, nous fûmes une pause restauration vers 21h30. Nous 
partageâmes les deux plateaux repas en trop avec les participants du fait de l’absence des 
deux intervenants. La soirée termina à 23h car nous apprîmes en arrivant pour les préparatifs 
que nous pouvions utiliser la salle jusqu’à minuit.
