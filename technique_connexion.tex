Afin que la soirée se déroule sans encombre il nous faut une connexion internet avec certains ports pour certains protocoles débloqués,

 et ce n'est pas une mince affaire. Tout d'abord nous avons demander l'avis de monsieur Mathieu et la procédure à suivre dans ce genre de cas.
Au fil des échanges nous avons donc fixé le lieu, le périmètre, la date, la durée, le nombre d'ordinateur et quelques autres points techniques.
Dans un premier temps, les tests se font dans notre salle de tp habituelle et notre intermédiaire pour les détails techniques est monsieur Eric Triquet.
Les premiers tests se font sur un seule machine, un ordinateur portable personnel, on utilise un adresse ip public et les tests sont plutôt satisfaisants.
Le déploiement vers la maison des étudiants pose problème. Le responsable sécurité ne veut pas ouvrir quatres adresses ip public vers des ordinateurs personnels,
 ce qui veut dire que nous devons penser à autre chose.
La solution à ce problème a été trouver en allant directement au CRI avec monsier Triquet afin de s'expliquer sur les besoins de la soirée.
Nous avons donc établi un terrain d'entente, la solution est d'utiliser un VLAN, c'est un réseau indépendant et isolé du reste du réseau et permet de garder
un contrôle sur les données transmises lors de notre soirée.
Les tests en salle tp se sont montrés concluants, et le déploiement sur la maison des étudiants n'a pas été long.
Les tests finaux sur site se sont montrés également concluant.
