Les prémices du projet

  L'objectif du projet de communication

    Les principaux objectifs du projet de communication sont :
      1. promouvoir la formation au travers d'activités tantot éducatives
      tantot lucrative.
      2. s'ouvrir au monde exterieur en montrant que des projets en dehors
      de l'informatique sont menées au sein de la licence
      proffessionnelle.
      3. Avoir les résponsabilitées qu'un organisateur/gestionnaire peut
      avoir ce qui peut entrainer une certaine pression.

  L'idée

    Ayant une passion commune autour de l'e-sport(expliquer) et des jeux
    vidéos en général, nous cherchions un projet ambitieux et original
    pour promouvoir la formation. Nous ne voulions pas d'un projet maintes
    & maintes fois réalisées et finalement peut motivant qui aurait
    reflété en nous l'obligation de faire plutôt que l'envie de faire.
    Ayant suivi avec attention la montée en puissance d'un nouveau
    phénomène qu'est le barcraft (faire réference sur la section du
    dessous) nous nous sommes mis en tête de présenter cette idée et de la
    défendre.

  Les Jeux considérés

    Lors de nos réunions nous avons eu plusieurs idée de jeux à presenté
    pendant la soirée. En effet, nous nous sommes dit que starcraft
    était peut être un segment trop étroit pour attirer du monde et
    qu'il fallait peut-être mieux présenté plusieurs jeux et proposer
    des plages horaires pour ceux-ci.

    - Starcraft : devait être présenté mais il n'y a qu'une personne du
    groupe qui à le jeu (qui est en plus décomposé en un jeu + une
    extention) ce qui fait que les deux autres devait l'acheter. De
    plus, brood war par son graphisme 2D aurait fait fuir de panique le
    public. celui-ci est présenté dans la section suivante (lien vers la
    presentation).

    - Starcraft II : Le jeu principal auquel tout les membres du groupe
    joue et connaissent bien. Celui-ci est décomposé en deux jeux distinct.
    La partie la plus interessante étant le jeu en ligne bien évidement.
    L'extension (heart of the swarm) étant sorti la veille de la soirée il a
    fallu nous empresser de l'acheter pour pouvoir y jouer et ainsi montrer
    les nouvelles unités et abilités de celles éxistante. Ce jeux est le
    premier auquel nous ayons pensé tout simplement parceque c'est celui-ci
    qui à démocratisé le concept de barcraft (lien explication barcraft)

    - League of Legends : (lien vers la presentation + abandonnée)

    - retro-gaming : On voulait proposer d'autre jeux de stratégies
    d'anthologies à côté des matchs commenté mais ça aurait dénaturer le
    projet pour finir en soirée jeu-vidéo ce qui n'a jamais été notre
    intention. De plus, il aurait fallu encadrer un nombre conséquend de
    materiel durant la soirée ce qui aurait été dur à gérer avec trois
    personnes.

    - jeu de plateau : Disposé de jeux de plateau se serait fait assez
    simplement. En effet, l'association *dès a la carte* organise ce genre
    d'événement assez régulièrement. Nous avons décidé assez vite de ne pas
    retenir cette idée car cela aurait causé trop de dispersion dans la
    soirée et le concept de barcraft n'aurait plus de sens.  (abandonnée
    assez-vite donc pas de presentation)

    Au final nous sommes revenu en arrières sur nos idées. Nous avons
    décidé de ne présenter que Starcraft II qui est le jeux que nous
    connaissons le plus. Cependant certains des jeux ont été encore
    proposés un mois avant la date buttoire, c'est pourquoi ceux-ci sont
    présenté dans ce rapport.

  La soumission de l'idée

    Nous avons cherché à démistifier la pratique du jeux-vidéos qui est
    soumis malgrès elle à un certain nombre d'idées recues & de
    préjugées. Il fallait mettre en avant l'aspect stratégique des jeux
    choisi et la proffessionalisation de ceux-ci qui est en
    voie. Cependant nous nous sommes heurté aux regles bien défini du
    projet de communication qui devait le plus possible ce déroulé loin
    de l'informatique et du monde numérique, autant dire que nous ne
    partions pas gagnant même si nous étions très motivé & déterminer a
    convaincre nos enseignants. Il y a eu des moments d'insertitudes et
    d'abandons parmis nous mais nous nous sommes efforcé de nous motiver
    les uns les autres. Nous avons du défendre à plusieurs reprises nos
    idées comparées aux autres projet ce qui nous à demandé d'être clair sur
    nos intentions.

  Projet accepté

    Notre projet à mis plus de temps que les autres à être validé.
    Celui-ci à été soumis à deux personnes. Nous nous sommes retrouvé
    pendant un laps de temps (2 semaines) sans savoir si notre projet était
    validé ou non. Nous étions inquiêt a son sujet parceque si il n'était
    pas accepté, cela nous aurait mis grandement en retard pour tout dabord
    en trouver un nouveau mais aussi le mettre sur pied. Nous avions pensé à
    plusieurs solution de remplacement mais celle-ci nous motivait pas
    autant voir pas du tout comparé à notre idée principale.

    Nous avions pensé à :
      - tournois de tennis de table : Deux membre du groupe en ont fait
    en club et aime ce sport cependant il est très difficile de se procuré
    du materiel pour tout un tournoi.
      - Post-it war : le concept est de proposer au gens des différents
    batiments de lille1 de déssiner avec des post-it de couleur. Les dessins
    aurait été soumis à des juges qui aurait décerné la meilleur oeuvre.
    Pour l'anecdote, cette idée à été réalité à la fin de cette année par
    l'association AEI (lien vers description)

    Heureusement pour nous, nous n'avons pas eu besoin d'y reflechir
    d'avantage et nous avons du nous mêttre au travail pour réaliser notre
    projet.

Présentation

    Jeux de stratégie en temps réel

      Une très bonne définition de ce qu'est le jeux de stratégie en
      temps réel est consultable sur wikipédia :

        Le jeu de stratégie en temps réel (STR, ou RTS pour la
        dénomination du genre en anglais : real-time strategy) est un type de
        jeu de stratégie particulier qui notamment et par opposition au jeu de
        stratégie au tour par tour n’utilise pas un découpage arbitraire du
        temps [...]

        Dans le tour par tour, donc, l’issue d’une confrontation est résolue par
        étapes, un combat en succédant un autre, de manière à laisser à chaque
        joueur le temps de réfléchir sur la prochaine étape. Une confrontation
        est alors résolue par combats successifs et isolés.

        Mais le problème qui se pose lorsqu’on veut simuler des affrontements
        réalistes est qu’en général, les unités n’attendent pas que ce soit leur
        tour pour attaquer et gérer l’affrontement de plusieurs unités en même
        temps et de façon réaliste relève du domaine de l’impossible pour l’être
        humain, jusqu’à l’invention des processeurs. Ceux-ci, programmés de
        manière à simuler un affrontement peuvent désormais gérer le déplacement
        de plusieurs unités simultanément et résoudre les combats de façon
        simultanée et rigoureuse dans le cadre d’un jeu vidéo.

    La licence Starcraft

      Historique

        Starcraft & Starcraft: Brood War

          Starcraft est un jeu vidéo de stratégie en temps réel (STR) développé
          par Blizzard Entertainment. Le premier opus est sorti en Europe le 31
          mars 1998 sur PC. Il s'inscrit dans la lignée des jeux de strategie de
          Blizzard maintenant très populaire auprès des amateurs du genre. Avec
          plus de 11 millions de copies vendues dans le monde, il est un des
          jeux vidéo sur PC les mieux vendus et reste à ce jour le jeu de
          stratégie en temps réel le plus vendu de tout les temps.

          (image)

        Starcraft 2 : (Wings of Liberty)

          Celui-ci sort dans le monde après 12 ans d'attente
          interminable le 27 juillet 2010. Blizzard ayant fait le choix marketing
          de faire trois campagnes distinctes, il y aura deux extensions de
          prevues ce qui à fait polémique à l'époque. Avec le deuxième opus,
          blizzard à fait le pari de démocratisé le jeux de stratégie en temps
          réel en rendant WoL plus facile d'accès comparées a Brood War qui lui
          requiérait beaucoup d'APM (nbp). Pari reussi puisqu'il connait
          rapidement un important succès commercial avec plus de 4,5 millions de
          copies vendues 6 mois après la sortie du jeu.

          (image)

        Starcraft 2 : (Heart of the Swarm)

          Heart of the Swarm est la première des deux extensions prévu
          du jeux Starcraft 2. Elle constitue la suite du premier chapitre Wings
          of Liberty. La date de sortie officielle est le 12 mars 2013. Cette opus
          apporte une nouvelle campagne et des nouvelles unités au mode
          multijoueur. Aucune statistique de vente n'a encore été publiée par
          blizzard actuellement.

        League of Legends

          - présentation rapide du jeu

	Le Barcraft

     Comme sous nom l'indique, il s'agit d'organiser une soirée autour
     du jeu starcraft avec l'ambiance d'un bar.	Pour cela, nous faisons
     appelle à des intervenants extérieurs. Afin d'ovrir un maximum de
     spectacle, nous avons demander à des joueurs professionnels ainsi
     que des commentateurs professionnels de participer à l'evénement.
     La soirée se déroule sous la forme de matchs commentés. Pour une
     bonne intéraction avec le public, les matchs sont vidéoprojetés sur
     un écran géant. Au cour de la soirée les joueurs du public pouront
     prendre place et jouer contre notre joueur professionnel.

Organisation

	Technique
		- Connexion internet avec certains ports ouverts
		- Premier test avec une adresse public en salle de tp concluant
		- Pour le deploiement vers la mde, le responsable du réseau ne voulait pas nous fournir 4 ip public
		- Mise au point d'une autre solution
		       - VLAN + NAT
		- Test en salle tp concluant
 		- Brassage des prises de la mde sur le VLAN
		- Test à la mde concluant

	Materiel
		- ordinateurs puissants
		- videoprojecteur
		- ampli + micro
		- tables et chaises
		- connectique
    - isolation des joueurs

	Salle et bar

    Nous cherchions un local qui aurait le maximum de materiel dont nous
    avions besoin pour la soirée pour plus de facilité. Nous avons tout
    dabord cherché a l'interieur du campus de lille avant de vouloir trouvé
    un commerce qui nous aurait demandé beaucoup plus d'éffort d'un point de
    vue recherche.

    Ayant participé a certaines soirées étudiantes qui se déroule le jeudi
    soir a la maison des étudiants (MDE), nous avons remarqué que celle-ci
    était disponible assez facilement, gratuitement et avec du materiel
    approprié. En effet, les thématiques proposées par la MDE sont très
    diverses. Elle peut aussi bien faire office de salle de reunion que de
    concert ou même de soirée cinéma.

    Cependant, pour avoir le droit d'organiser quelquechose dans celle-ci il
    faut appartenir à une association étudiante connu du campus. Aucun
    membre du groupe étant licencé dans ce genre d'initiative nous étions un
    peu inquiet. Après s'être renseigné a differents endroits, il faut
    simplement se rapprocher d'une association pour qu'elle puisse se porte
    garant de l'évenement.

    La nécéssité d'avoir une association derrière nous est obligatoire pour
    l'obtention d'une extension d'assurance pour les biens et les personnes
    au cas ou un problème majeur surviendrait lors de la soirée.

    - redaction de la convention avec l'aei
    - L'AMUL
    - Plusieurs mails et rendez-vous

	Partenariat et sponsor

    - Demande de financement du projet auprès de *JE SAIS PLUS QUOI*

      Nous nous sommes tout dabord rendu au batiment *JE SAIS PLUS QUOI*
      pour récupéré un dossier de demande de financement pour les projets des
      étudiants. Ce dossier requierait un certain nombre de pièces
      justificative comme le planning de déroulement de la soirée, le montant
      envisagé ainsi que le potentiel nombre de participant. Par la suite un
      entretiens avec les membres du groupe face à un jury de *JE SAIS
      PLUS QUOI*. Nous avons abandonnée cette possibilité car nous ne savions
      pas si le projet était possible (connexion).

    - collaboration avec l'aei

      Nous avons choisi de traiter avec l'AEI (association des étudiants
      en informatique) qui sont des habituées de ce genre de soirée à
      thématique *geek* et emprunteur historique de la maison des étudiants.
      Leur principal local ce situe dans le batiment M3 au milieu de la
      cité scientifique ou sont concentré la pluspart des informaticiens de
      lille1.

    - demande de financement auprès de Micromania (v2)

      Une demande de financement à été faite auprès du célèbre magasin de
      jeux-vidéo Micromania qui nous a vallu un refus cathégorique. Cette
      enseigne ne sponsorisant pas d'évenement par soucis
      budgétaire. (j'invente)

    - demande de sponsor Materiel.net (lomme)

    - envoie d'un mail au service communication sans reponse

    - la licence pro, paiement des repas des intervenants

      N'ayant trouvé aucun moyen de finacement pour permettre aux
      intervenants et nous même de nous restaurer lors de l'évenement, nous
      n'avions pas d'autre choix que de répartir les coûts entre membre du
      projet. C'est à ce moment la que le réponsable de la formation DA2I
      proposa de débloqué les fonds allouées à la licence pour permettre de
      financer les évnements des étudiants. Ce fut une aubaine pour nous.

      Nous avons tout dabord pensé à acheter nous même les aliments pour
      ensuite ce faire rembourser avec cette somme mais cela posait des
      problèmes logistiques. Une autre idée à donc été proposé : commander des
      plateaux repas (qui est un service proposé par l'iut). Il nous a donc
      été demandé de réunir des informations auprès des intervenants de la
      soirée, comme leur nom, prenom, adresse, fonction pour que cette demande
      aboutisse.

	Intervenants

    Leur role dans la soirée

      Nous souhaitions invité des personnes exterieures pour le projet à
      differente fins. Il nous fallait un ou des commentateurs pour les
      parties ainsi que des joueurs. Ceux-ci, pour engrangé du monde, devait
      être connu de la scène française.


    - Alexandre "Makoz" Chilling
      - Voir les échanges facebook

    Un des membres du groupe qui, connaissant bien la scène française
    autour de starcraft 2, s'est occupé de contacter des intervenants. Le
    role

		- team *aAa* demande 500€
                    - pas de négociation possible
                    - somme > budget prévisionnel
                    - refus de l'offre
    - Thomas Brandt aka soey & Julien roy (lieu dunkerque) (demande 0€)
        - Acceptation de l'offre

        Communication
                - Mail general a tous les etudiants (newslettre mde)
                - affiche A3, imprimé à l'IUT (30 unités disposées dans la fac)
                - questionnaire diffusé à tous les étudiants
                - Facebook : création d'un evenement (MORT)

        Bilan financier
                Prévisionnel
                    - salle
                    - bar
                    - matos
                    - intervenant
                    - comm
                    - repas
                Reel
                   ...

        Deroulement de la soirée
                Previsionnel
                    18h/22h
                    arrivé 16h
                    depart 23h
                    deroulement
                        - videos d'introduction au jeu pour les neophytes
                        - parties commentées avec des joueurs professionnels
                        - parties commentées avec des joueurs du public

                Imprevu
                   Problème de connexion battle.net
                   Meteo defavorable
                   Intervenants pas présent
                   impac sur le déroulement prévisionnel

                Réel
                    18h/23h20
                    arrivé 16h15
                    départ 23h55
                    deroulement
                        - videos d'untroduction au jeu pour les neophytes
                        - parties commentées par julien avec le public
                        - coupure pour manger vers 21h pendant 20 minutes

Bilan
