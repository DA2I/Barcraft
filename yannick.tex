Les prémices du projet

  L'objectif du projet de communication

    - promouvoir la formation (fun)
    - s'ouvrir au monde exterieur
    - avoir des responsabilitées
    - autres

  L'idée

    Ayant une passion commune autour de l'e-sport(expliquer) et des jeux
    vidéos en général, nous cherchions un projet ambitieux et original
    pour promouvoir la formation. Nous ne voulions pas d'un projet maintes
    & maintes fois réalisées et finalement peut motivant qui aurait
    reflété en nous l'obligation de faire plutôt que l'envie de faire.
    Ayant suivi avec attention la montée en puissance d'un nouveau
    phénomène qu'est le barcraft (faire réference sur la section du
    dessous) nous nous sommes mis en tête de présenter cette idée et de la
    défendre.

  Les Jeux considérés

    Lors de nos réunions nous avons eu plusieurs idée de jeux à presenté
    pendant la soirée. En effet, nous nous sommes dit que starcraft
    était peut être un segment trop étroit pour attirer du monde et
    qu'il fallait peut-être mieux présenté plusieurs jeux et proposer
    des plages horaires pour ceux-ci au risque de nous éparpillé.

    - broodwar (lien vers la presentation + abandonnée)
    - sc2 (lien vers la presentation)
    - lol (lien vers la presentation + abandonnée)
    - retro-gaming (abandonnée assez-vite donc pas de presentation)
    - jeu de plateau (abandonnée assez-vite donc pas de presentation)

  La soumission de l'idée

    Nous avons cherché à démistifier la pratique du jeux-vidéos qui est
    soumis malgrès elle à un certain nombre d'idées recues & de
    préjugées. Il fallait mettre en avant l'aspect stratégique des jeux
    choisi, le temps et la proffessionalisation de ceux-ci qui est en
    voie. Cependant nous nous sommes heurté aux regles bien défini du
    projet de communication qui devait le plus possible ce déroulé loin
    de l'informatique et du monde numérique, autant dire que nous ne
    partions pas gagnant même si nous étions très motivé & déterminer a
    convaincre nos enseignants. Il y a eu des moments d'insertitudes et
    d'abandons parmis nous mais nous nous sommes efforcé de nous motiver
    les uns les autres.

  Le feu vert (à changer)

      (à developper)

Présentation

    La licence Starcraft

      Historique

        Starcraft & Starcraft: Brood War

          Starcraft est un jeu vidéo de stratégie en temps réel (STR) développé
          par Blizzard Entertainment. Le premier opus est sorti en Europe le 31
          mars 1998 sur PC. Il s'inscrit dans la lignée des jeux de strategie de
          Blizzard maintenant très populaire auprès des amateurs du genre. Avec
          plus de 11 millions de copies vendues dans le monde, il est un des
          jeux vidéo sur PC les mieux vendus et reste à ce jour le jeu de
          stratégie en temps réel le plus vendu de tout les temps.

        Starcraft 2 (wol & hots)

          - présentation rapide du jeu

        League of Legends

          - présentation rapide du jeu

	Le Barcraft

     Comme sous nom l'indique, il s'agit d'organiser une soirée autour
     du jeu starcraft avec l'ambiance d'un bar.	Pour cela, nous faisons
     appelle à des intervenants extérieurs. Afin d'ovrir un maximum de
     spectacle, nous avons demander à des joueurs professionnels ainsi
     que des commentateurs professionnels de participer à l'evénement.
     La soirée se déroule sous la forme de matchs commentés. Pour une
     bonne intéraction avec le public, les matchs sont vidéoprojetés sur
     un écran géant. Au cour de la soirée les joueurs du public pouront
     prendre place et jouer contre notre joueur professionnel.

Organisation

	Technique
		- Connexion internet avec certains ports ouverts
		- Premier test avec une adresse public en salle de tp concluant
		- Pour le deploiement vers la mde, le responsable du réseau ne voulait pas nous fournir 4 ip public
		- Mise au point d'une autre solution
		       - VLAN + NAT
		- Test en salle tp concluant
 		- Brassage des prises de la mde sur le VLAN
		- Test à la mde concluant

	Materiel
		- ordinateurs puissants
		- videoprojecteur
		- ampli + micro
		- tables et chaises
		- connectique
    - isolation des joueurs

	Salle et bar
	      - Maison des etudiants
	      	       - redaction de la convention avec l'aei
		       - extensions d'assurance pour les biens et les personnes
	      - L'AMUL
			- Plusieurs mails et rendez-vous

	Partenariat et sponsor
        - Demande de financement du projet auprès de *JE SAIS PLUS QUOI*
		    - collaboration avec l'aei
        - demande de financement auprès de Micromania (v2)
		    - demande de sponsor Materiel.net (lomme)
		    - envoie d'un mail au service communication sans reponse
		    - la licence pro, paiement des repas des intervenants

	Intervenants
    - Alexandre "Makoz" Chilling
      - Voir les échanges facebook
		- team *aAa* demande 500€
                    - pas de négociation possible
                    - somme > budget prévisionnel
                    - refus de l'offre
    - Thomas Brandt aka soey & Julien roy (lieu dunkerque) (demande 0€)
        - Acceptation de l'offre

        Communication
                - Mail general a tous les etudiants (newslettre mde)
                - affiche A3, imprimé à l'IUT (30 unités disposées dans la fac)
                - questionnaire diffusé à tous les étudiants
                - Facebook : création d'un evenement (MORT)

        Bilan financier
                Prévisionnel
                    - salle
                    - bar
                    - matos
                    - intervenant
                    - comm
                    - repas
                Reel
                   ...

        Deroulement de la soirée
                Previsionnel
                    18h/22h
                    arrivé 16h
                    depart 23h
                    deroulement
                        - videos d'introduction au jeu pour les neophytes
                        - parties commentées avec des joueurs professionnels
                        - parties commentées avec des joueurs du public

                Imprevu
                   Problème de connexion battle.net
                   Meteo defavorable
                   Intervenants pas présent
                   impac sur le déroulement prévisionnel

                Réel
                    18h/23h20
                    arrivé 16h15
                    départ 23h55
                    deroulement
                        - videos d'untroduction au jeu pour les neophytes
                        - parties commentées par julien avec le public
                        - coupure pour manger vers 21h pendant 20 minutes

Bilan
