\subsection{Technique}%{{{
\label{sub:technique}

%TODO
Technique
  - Connexion internet avec certains ports ouverts
  - Premier test avec une adresse public en salle de tp concluant
  - Pour le deploiement vers la mde, le responsable du réseau ne voulait pas nous fournir 4 ip public
  - Mise au point d'une autre solution
         - VLAN + NAT
  - Test en salle tp concluant
  - Brassage des prises de la mde sur le VLAN
  - Test à la mde concluant

% subsection technique (end)%}}}
\subsection{Materiel}%{{{
\label{sub:materiel}

Pour que la soirée se déroule dans les règles de l'art, il nous fallait
du materiel spécifique. En effet, comme expliqué dans la section
barcraft (link) des ordinateurs, une connexion à internet et un
vidéo-projecteur sont requis (et de la bière si possible) pour prétendre
organiser ce type d'evenement.

- ordinateurs puissants

  Starcraft 2 est un jeu assez reçent comme nous l'avons vu dans la
partie (link) ce qu'il fait qu'il requiert des machines puissantes pour jouer
dans bonnes conditions. Nous avons tout d'abord cherché à emprunté des
machines auprès de l'AEI mais celle-ci - à part pour faire tourner brood
war - sont mauvaises étant donnée qu'elle ne contienne qu'un chipset
graphique (explication). Par la suite nous avons discuté avec les
intervenants pour que ceux-ci ramènent leurs configurations. Ça nous
aurait permis de nous soulager un peu mais aussi d'améliorer la qualité
de jeux du joueur professionnel. Il est connu que l'on joue mieux sur sa
machine - avec laquel on à l'habitude - plutôt que la machine d'un
autre.

- videoprojecteur
- ampli + micro
- tables et chaises
- connectique

Les autres materiaux dont nous avions besoin comme le videoprojecteur,
les micros, la connectique et les tables/chaises était disponible directement à
la maison des étudiants ce qui nous a grandement facilité la tâche.

- isolation des joueurs

Pour ce qui est de l'isolation des joueurs, nous avons décidé un peu à
la dernière minute qu'un casque de chantier et des écouteurs
intra-oriculaires suffirait. C'est ce qui est utilisé la pluspart du
temps sauf pour les gros évenements qui eux construisent des "box
insonorisées". Étant donnée que notre projet n'est pas de la même
envergure nous nous en sommes passé bien évidemment.

% subsection materiel (end)%}}}
\subsection{Salle et bar} %{{{
\label{sub:salle_et_bar}

Nous cherchions un local qui aurait le maximum de materiel dont nous
avions besoin pour la soirée pour plus de facilité. Nous avons tout
dabord cherché a l'interieur du campus de lille avant de vouloir trouvé
un commerce qui nous aurait demandé beaucoup plus d'éffort d'un point de
vue recherche.

Ayant participé a certaines soirées étudiantes qui se déroule le jeudi
soir a la maison des étudiants (MDE), nous avons remarqué que celle-ci
était disponible assez facilement, gratuitement et avec du materiel
approprié. En effet, les thématiques proposées par la MDE sont très
diverses. Elle peut aussi bien faire office de salle de reunion que de
concert ou même de soirée cinéma.

Cependant, pour avoir le droit d'organiser quelquechose dans celle-ci il
faut appartenir à une association étudiante connu du campus. Aucun
membre du groupe étant licencé dans ce genre d'initiative nous étions un
peu inquiet. Après s'être renseigné a differents endroits, il faut
simplement se rapprocher d'une association pour qu'elle puisse se porte
garant de l'évenement.

La nécéssité d'avoir une association derrière nous est obligatoire pour
l'obtention d'une extension d'assurance pour les biens et les personnes
au cas ou un problème majeur surviendrait lors de la soirée.

- redaction de la convention avec l'aei
- L'AMUL
- Plusieurs mails et rendez-vous

% subsection salle_et_bar (end)%}}}
\subsection{Partenariat et sponsor} %{{{
\label{sub:partenariat_et_sponsor}

  %TODO
- Demande de financement du projet auprès de *JE SAIS PLUS QUOI*

  Nous nous sommes tout dabord rendu au batiment A3
  pour récupéré un dossier de demande de financement pour les projets des
  étudiants. Ce dossier requierait un certain nombre de pièces
  justificative comme le planning de déroulement de la soirée, le montant
  envisagé ainsi que le potentiel nombre de participant. Par la suite un
  entretiens avec les membres du groupe face à un jury de *JE SAIS
  PLUS QUOI*. Nous avons abandonnée cette possibilité car nous ne savions
  pas si le projet était possible (connexion).

  %TODO
- collaboration avec l'aei

  Nous avons choisi de traiter avec l'AEI (association des étudiants
  en informatique) qui sont des habituées de ce genre de soirée à
  thématique *geek* et emprunteur historique de la maison des étudiants.
  Leur principal local ce situe dans le batiment M5 au milieu de la
  cité scientifique ou sont concentré la pluspart des informaticiens de
  lille1.
  Nous avons donc

  %TODO
- demande de financement auprès de Micromania (v2)

  Une demande de financement à été faite auprès du célèbre magasin de
  jeux-vidéo Micromania qui nous a vallu un refus cathégorique. Cette
  enseigne ne sponsorisant pas d'évenement par soucis
  budgétaire. (j'invente)

  %TODO
- demande de sponsor Materiel.net (lomme)

  %TODO
- envoie d'un mail au service communication sans reponse

  N'ayant trouvé aucun moyen de finacement pour permettre aux
  intervenants et nous même de nous restaurer lors de l'évenement, nous
  n'avions pas d'autre choix que de répartir les coûts entre membre du
  projet. C'est à ce moment la que le réponsable de la formation DA2I
  proposa de débloqué les fonds allouées à la licence pour permettre de
  financer les évnements des étudiants. Ce fut une aubaine pour nous.

  Nous avons tout dabord pensé à acheter nous même les aliments pour
  ensuite ce faire rembourser avec cette somme mais cela posait des
  problèmes logistiques. Une autre idée à donc été proposé : commander des
  plateaux repas (qui est un service proposé par l'iut). Il nous a donc
  été demandé de réunir des informations auprès des intervenants de la
  soirée, comme leur nom, prenom, adresse, fonction pour que cette demande
  aboutisse.

% subsection partenariat_et_sponsor (end)%}}}
\subsection{Intervenants}%{{{
\label{sub:intervenants}

- Leur role dans la soirée

  Nous souhaitions invité des personnes exterieures pour le projet à
  differente fins. Il nous fallait un ou des commentateurs pour les
  parties ainsi que des joueurs. Ceux-ci, pour engrangé du monde, devait
  être connu de la scène française.

- team *aAa* demande 500€
                - pas de négociation possible
                - somme > budget prévisionnel
                - refus de l'offre

Une deuxième demande à été faite auprès du clan *aAa* qui comporte
plusieurs équipes professionnelles dont une sur Starcraft 2 et une sur
League of Legends. Les négociations se sont déroulé par échange de mail
interposé.  Ceux-ci par l'intermediaire de leur manager ayant pour
pseudo *zidwait* fut interessé par la proposition. Ils nous ont proposé
quatres personnes, deux commentateurs et deux joueurs situés de toute
part en France. Cependant, l'évenement n'étant pas considéré comme un
tournoi mais comme une prestation de service par leur sponsors, ceux-ci
n'ont pas voulu prendre en charge les frais de déplacement. La somme
requise pour faire venir les personnes consernées était éstimé à hauteur
de 500 euros ce qui de loin dépasse notre budget prévisionnel. Nous
avons donc été contraint de refuser cette offre. Il nous ont par contre
proposé de retransmettre la soirée sur leur stream si nous filmions
l'évennement. Nous avons refusé pour plusieurs raisons. Tout dabord cela
aurait monopolisé l'un d'entre nous toute la soirée, ensuite cela aurait
fourni du contenu pour leur stream gratuitement alors qu'il n'ont
participé d'aucune manière au projet. Enfin, personne dans le groupe
n'as pour désir de devenir caméraman.

- Alexandre "Makoz" Chilling
  - Voir les échanges facebook

Un des membres du groupe qui, connaissant bien la scène française autour
de starcraft 2, s'est occupé de contacter des intervenants.  Celui-ci,
appréciant Alexandre "maKoZ" Chilling la contacté en premier.  maKoz qui
était (au moment de la demande) manager de *spart legion*, une equipe
française faisant de plus en plus parlé d'elle. En plus de ses qualitées
managériales il est commentateur de matchs via des streams (explication)
et propose de decortiquer des matchs entre joueurs professionnelle sur
sa page youtube. Malheureusement, Alexandre par ses responsabilités n'a
pas pu ce joindre à nous.

- Thomas Brandt aka soey et Julien roy (lieu dunkerque) (demande 0€)
    - Acceptation de l'offre

%TODO
Par la suite nous avons exposé nos problèmes pour trouver des
intervenants à Makoz. Celui-ci nous a conseillé de contacter Thomas
"Soey" Brandt qui est un joueur semi-professionnel sur Lille. Thomas a
été séduit par notre offre qu'il a accepté assez vite lorsque nous lui
avons présenté le projet.

% subsection intervenants (end)%}}}
\subsection{Communication}%{{{
\label{sub:communication}

Communication
        - Mail general a tous les etudiants (newslettre mde)
        - affiche A3, imprimé à l'IUT (30 unités disposées dans la fac)
        - questionnaire diffusé à tous les étudiants
        - Facebook : création d'un evenement (MORT)

% subsection communication (end)%}}}
\subsection{Bilan financier}%{{{
\label{sub:bilan_financier}

  %TODO
Bilan financier
        Prévisionnel
            - salle
            - bar
            - matos
            - intervenant
            - comm
            - repas
        Reel
           ...

% subsection bilan_financier (end)%}}}
\subsection{Bilan de la soirée}%{{{
\label{sub:Bilan_de_la_soiree}

Previsionnel
    18h/22h
    arrivé 16h
    depart 23h
    deroulement
        - videos d'introduction au jeu pour les neophytes
        - parties commentées avec des joueurs professionnels
        - parties commentées avec des joueurs du public

Imprevu
   Problème de connexion battle.net
   Meteo defavorable
   Intervenants pas présent
   impac sur le déroulement prévisionnel

Réel
    18h/23h20
    arrivé 16h15
    départ 23h55
    deroulement
        - videos d'untroduction au jeu pour les neophytes
        - parties commentées par julien avec le public
        - coupure pour manger vers 21h pendant 20 minutes

Previsionnel :

La date de la soirée devait être judicieusement choisie afin d'obtenir une affluence satisfaisante.

Ayant vu grâce au mailing de l’université que la MDE accueillait des soirées chaque jeudi
nommées « Jeudifférents », nous nous sommes dit qu’une soirée comme la notre collait
parfaitement à appellation. Malheureusement, nous avions commencé à prendre contact
début février pour la réservation, ce qui fut trop tard. Tous les jeudis sur deux mois étaient
déjà réservés.
Il nous a donc fallu choisir un autre jour de la semaine. Après renseignements auprès de la
MDE, seuls le mercredi et le vendredi étaient libres la semaine du 11 au 17 mars. Le choix ne
fut pas difficile, le mercredi fut choisi naturellement car les étudiants ayant un appartement
dans la métropole rentrent chez eux le vendredi. De plus, le mercredi 13 mars correspondait
au lendemain de la sortie de la nouvelle extension du jeu.
A propos de l’horaire de la soirée, nous avions décidé de commencer dès 18h pour finir à
22h. En effet, le fait de commencer à 18h permet aux étudiants de se rendre à la soirée juste
après les cours. Ensuite, d’après l’AEI, la MDE ferme à 23h et pour ouvrir au-delà de cette
heure, il était nécessaire d’avoir une dérogation pénible à obtenir. Nous nous sommes donc
donné une heure afin d’avoir le temps de remballer, mais également de partir plus tôt pour
reconduire les intervenants chez eux, à Dunkerque.
Après s’être mis d’accord sur la date et l’heure de la soirée, nous devions discuter du
programme.
Il ne fut pas aisé de définir un programme à heures fixes, car les parties n’ont pas une durée
définie. Elles peuvent durer entre 5 et 50 minutes, la moyenne se situant à 25 minutes de jeu.
C’est pourquoi nous avons décidé d’énoncer le contenu de la soirée sans y faire figurer
d’heure.
Nous avions donc prévu de commencer par 20 minutes d’introduction en vidéo sur le jeu et
ses mécanismes afin de pouvoir rendre la soirée plus compréhensible pour les néophytes.
Puis enchainer avec les matchs commentés des intervenants pour finir avec des matchs
confrontant des joueurs du public. Nous avons choisis cet ordre car nous pensions que le plus
amusant était de faire jouer le public, et que cela permettrait de les retenir un maximum.
Imprévus majeurs :
Arrivé le jour J, nous avons été confrontés à plusieurs imprévus.
La météo ne fut pas clémente le 12 mars ainsi que le 13 mars, la neige étant fortement
tombée ces jours là. La circulation était devenue difficile dans la région et beaucoup de cours
furent perturbés.
Ceci impacta fortement l’affluence de la soirée, et surtout son déroulement. Les intervenants
ne purent pas venir du fait des conditions climatiques. Le retour à la fin de la soirée aurait été
dangereux à cause du verglas, et aucun hébergement n’eut pu être possible.
Ces imprévus nous ont donc forcés à improviser.

La soirée :

%TODO On a du ramener nos pc !

Les préparatifs commencèrent à 16h, tout se passa bien ou presque. Nous nous aperçûmes
qu’il fallût utiliser une connectique autre que VGA afin de connecter nos PC au
vidéoprojecteur de la MDE. Heureusement, le directeur de la MDE M.Bross possédait un
vidéoprojecteur VGA avec une bonne qualité de projection.
Une dizaine de personnes arrivèrent passé 18h. Comme prévu à cause du climat, peu de
personnes assistèrent à la soirée. Nous avons recensé une vingtaine de personnes
différentes, mais jamais plus de 11 personnes simultanément.
A cause des imprévus et compte tenu de la faible affluence, nous avons décidé de faire jouer
directement les personnes présentes en projetant leurs parties sur vidéoprojecteur comme
prévu. Julien étant parmi nous le plus expérimenté, il put donc commenter quelques matchs
au cours de la soirée.
Vers 18h50, nous fûmes encore victime d’un imprévu. Les serveurs de jeu de tous les jeux en
ligne de Blizzard tombèrent, nous empêchant de jouer pendant 30 minutes. Pendant ce
temps, nous diffusions les vidéos explicatives de Starcraft II, ce qui permit de combler le
blanc engendré. Néanmoins, un groupe de 4 personnes s’en alla durant ce laps de temps.
Une fois la situation rentrée dans l’ordre, nous pûmes continuer la soirée en petit comité et
terminer à 23h. Entre temps, nous fûmes une pause restauration vers 21h30. Nous
partageâmes les deux plateaux repas en trop avec les participants du fait de l’absence des
deux intervenants. La soirée termina à 23h car nous apprîmes en arrivant pour les préparatifs
que nous pouvions utiliser la salle jusqu’à minuit.

% subsection Bilan_de_la_soiree (end)%}}}
