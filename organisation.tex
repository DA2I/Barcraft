\subsection{Technique}%{{{
\label{sub:technique}

Technique
  - Connexion internet avec certains ports ouverts
  - Premier test avec une adresse public en salle de tp concluant
  - Pour le deploiement vers la mde, le responsable du réseau ne voulait pas nous fournir 4 ip public
  - Mise au point d'une autre solution
         - VLAN + NAT
  - Test en salle tp concluant
  - Brassage des prises de la mde sur le VLAN
  - Test à la mde concluant

% subsection technique (end)%}}}
\subsection{Materiel}%{{{
\label{sub:materiel}

- ordinateurs puissants
- videoprojecteur
- ampli + micro
- tables et chaises
- connectique
- isolation des joueurs

% subsection materiel (end)%}}}
\subsection{Salle et bar} %{{{
\label{sub:salle_et_bar}

Nous cherchions un local qui aurait le maximum de materiel dont nous
avions besoin pour la soirée pour plus de facilité. Nous avons tout
dabord cherché a l'interieur du campus de lille avant de vouloir trouvé
un commerce qui nous aurait demandé beaucoup plus d'éffort d'un point de
vue recherche.

Ayant participé a certaines soirées étudiantes qui se déroule le jeudi
soir a la maison des étudiants (MDE), nous avons remarqué que celle-ci
était disponible assez facilement, gratuitement et avec du materiel
approprié. En effet, les thématiques proposées par la MDE sont très
diverses. Elle peut aussi bien faire office de salle de reunion que de
concert ou même de soirée cinéma.

Cependant, pour avoir le droit d'organiser quelquechose dans celle-ci il
faut appartenir à une association étudiante connu du campus. Aucun
membre du groupe étant licencé dans ce genre d'initiative nous étions un
peu inquiet. Après s'être renseigné a differents endroits, il faut
simplement se rapprocher d'une association pour qu'elle puisse se porte
garant de l'évenement.

La nécéssité d'avoir une association derrière nous est obligatoire pour
l'obtention d'une extension d'assurance pour les biens et les personnes
au cas ou un problème majeur surviendrait lors de la soirée.

- redaction de la convention avec l'aei
- L'AMUL
- Plusieurs mails et rendez-vous

% subsection salle_et_bar (end)%}}}
\subsection{Partenariat et sponsor} %{{{
\label{sub:partenariat_et_sponsor}

- Demande de financement du projet auprès de *JE SAIS PLUS QUOI*

  Nous nous sommes tout dabord rendu au batiment *JE SAIS PLUS QUOI*
  pour récupéré un dossier de demande de financement pour les projets des
  étudiants. Ce dossier requierait un certain nombre de pièces
  justificative comme le planning de déroulement de la soirée, le montant
  envisagé ainsi que le potentiel nombre de participant. Par la suite un
  entretiens avec les membres du groupe face à un jury de *JE SAIS
  PLUS QUOI*. Nous avons abandonnée cette possibilité car nous ne savions
  pas si le projet était possible (connexion).

- collaboration avec l'aei

  Nous avons choisi de traiter avec l'AEI (association des étudiants
  en informatique) qui sont des habituées de ce genre de soirée à
  thématique *geek* et emprunteur historique de la maison des étudiants.
  Leur principal local ce situe dans le batiment M3 au milieu de la
  cité scientifique ou sont concentré la pluspart des informaticiens de
  lille1.

- demande de financement auprès de Micromania (v2)

  Une demande de financement à été faite auprès du célèbre magasin de
  jeux-vidéo Micromania qui nous a vallu un refus cathégorique. Cette
  enseigne ne sponsorisant pas d'évenement par soucis
  budgétaire. (j'invente)

- demande de sponsor Materiel.net (lomme)

- envoie d'un mail au service communication sans reponse

- la licence pro, paiement des repas des intervenants

  N'ayant trouvé aucun moyen de finacement pour permettre aux
  intervenants et nous même de nous restaurer lors de l'évenement, nous
  n'avions pas d'autre choix que de répartir les coûts entre membre du
  projet. C'est à ce moment la que le réponsable de la formation DA2I
  proposa de débloqué les fonds allouées à la licence pour permettre de
  financer les évnements des étudiants. Ce fut une aubaine pour nous.

  Nous avons tout dabord pensé à acheter nous même les aliments pour
  ensuite ce faire rembourser avec cette somme mais cela posait des
  problèmes logistiques. Une autre idée à donc été proposé : commander des
  plateaux repas (qui est un service proposé par l'iut). Il nous a donc
  été demandé de réunir des informations auprès des intervenants de la
  soirée, comme leur nom, prenom, adresse, fonction pour que cette demande
  aboutisse.

% subsection partenariat_et_sponsor (end)%}}}
\subsection{Intervenants}%{{{
\label{sub:intervenants}

Leur role dans la soirée

  Nous souhaitions invité des personnes exterieures pour le projet à
  differente fins. Il nous fallait un ou des commentateurs pour les
  parties ainsi que des joueurs. Ceux-ci, pour engrangé du monde, devait
  être connu de la scène française.


- Alexandre "Makoz" Chilling
  - Voir les échanges facebook

Un des membres du groupe qui, connaissant bien la scène française
autour de starcraft 2, s'est occupé de contacter des intervenants. Le
role

- team *aAa* demande 500€
                - pas de négociation possible
                - somme > budget prévisionnel
                - refus de l'offre
- Thomas Brandt aka soey & Julien roy (lieu dunkerque) (demande 0€)
    - Acceptation de l'offre


% subsection intervenants (end)%}}}
\subsection{Communication}%{{{
\label{sub:communication}

Communication
        - Mail general a tous les etudiants (newslettre mde)
        - affiche A3, imprimé à l'IUT (30 unités disposées dans la fac)
        - questionnaire diffusé à tous les étudiants
        - Facebook : création d'un evenement (MORT)

% subsection communication (end)%}}}
\subsection{Bilan financier}%{{{
\label{sub:bilan_financier}

Bilan financier
        Prévisionnel
            - salle
            - bar
            - matos
            - intervenant
            - comm
            - repas
        Reel
           ...

% subsection bilan_financier (end)%}}}
\subsection{Déroulement de la soirée}%{{{
\label{sub:deroulement_de_la_soiree}

        Previsionnel
            18h/22h
            arrivé 16h
            depart 23h
            deroulement
                - videos d'introduction au jeu pour les neophytes
                - parties commentées avec des joueurs professionnels
                - parties commentées avec des joueurs du public

        Imprevu
           Problème de connexion battle.net
           Meteo defavorable
           Intervenants pas présent
           impac sur le déroulement prévisionnel

        Réel
            18h/23h20
            arrivé 16h15
            départ 23h55
            deroulement
                - videos d'untroduction au jeu pour les neophytes
                - parties commentées par julien avec le public
                - coupure pour manger vers 21h pendant 20 minutes

% subsection deroulement_de_la_soiree (end)%}}}
