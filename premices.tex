\subsection{L'objectif du projet de communication}%{{{
\label{sub:l_objectif_du_projet_de_communication}

Les principaux objectifs du projet de communication sont : 1. Promouvoir
la formation au travers d'activités tantôt éducatives tantôt lucrative.
2. S'ouvrir au monde extérieur en montrant que des projets en dehors de
l'informatique sont menées au sein de la licence professionnelle.  3.
Faire face a des responsabilités et à la pression que celle-ci peut
entrainer.

% subsection l_objectif_du_projet_de_communication (end)%}}}
\subsection{L'idée}%{{{
\label{sub:l_idee}

Ayant une passion commune autour de l'e-sport(expliquer) et des jeux
vidéos en général, nous cherchions un projet ambitieux et original
pour promouvoir la formation. Nous ne voulions pas d'un projet maintes
& maintes fois réalisées et finalement peut motivant qui aurait
reflété en nous l'obligation de faire plutôt que l'envie de faire.
Ayant suivi avec attention la montée en puissance d'un nouveau
phénomène qu'est le Barcraft (faire réference sur la section du
dessous) nous nous sommes mis en tête de présenter cette idée et de la
défendre.

% subsection l_idee (end) %}}}
\subsection{Les jeux considérés}%{{{
\label{sub:les_jeux_consideres}

Lors de nos réunions nous avons eu plusieurs idée de jeux à présenter
pendant la soirée. En effet, nous nous sommes dit que Starcraft (lien
desc) était peut être un segment trop étroit pour attirer du monde et
qu'il fallait mieux présenté plusieurs jeux et proposer des plages
horaires pour ceux-ci. Nous allons voir plus loin que cette idée ne fut
pas retenu au final.

- Starcraft : devait être présenté mais il n'y a qu'une personne du
groupe qui à le jeu (qui est en plus décomposé en un jeu + une
extension) ce qui fait que les deux autres devait l'acquerir. De
plus, Brood War par son graphisme 2D aurait fait fuir de panique le
public. Celui-ci est présenté dans la section suivante (lien vers la
presentation).

- Starcraft II : Le jeu principal auquel tout les membres du groupe joue
et connaissent bien. Celui-ci est décomposé en deux jeux distinct.  La
partie la plus intéressante étant le jeu en ligne bien évidement.
L'extension (Heart of the Swarm) étant sorti la veille de la soirée il a
fallu nous empresser de l'acheter pour pouvoir y jouer et ainsi montrer
les nouvelles unités du jeu. Ce jeux est le premier auquel nous ayons
pensé tout simplement parce que c'est celui-ci qui à démocratisé le
concept de Barcraft (lien explication barcraft)

%TODO
- League of Legends : (lien vers la presentation + abandonnée)

- retro-gaming : On voulait proposer d'autre jeux de stratégies
d'anthologies à côté des matchs commenté mais ça aurait dénaturer le
projet pour finir en soirée jeux vidéos ce qui n'a jamais été notre
intention. De plus, il aurait fallu encadrer un nombre conséquent de
matériel durant la soirée ce qui aurait été dur à gérer à trois
personnes.

- jeu de plateau : Nous avons pensé à nous procurer des jeux de
plateaux. En effet, l'association *dès a la carte* organise ce genre
d'événement assez régulièrement. Nous avons décidé assez vite de ne pas
retenir cette idée car cela aurait causé trop de dispersion dans la
soirée et le concept de Barcraft n'aurait plus de sens.

Au final nous sommes revenu en arrières sur nos idées. Nous avons décidé
de ne présenter que Starcraft II qui est le jeux que nous connaissons le
plus. Cependant certains des jeux ont été encore proposés un mois avant
la date buttoir, c'est pourquoi ceux-ci sont présenté dans ce rapport.


% subsection les_jeux_consideres (end)%}}}
\subsection{La soumission de l'idée}%{{{
\label{sub:la_soumission_de_leidee}

Nous avons cherché à démystifier la pratique du jeux vidéos qui est
soumis malgré elle à un certain nombre d'idées reçues et de préjugées.
Il fallait mettre en avant l'aspect stratégique des jeux choisi et la
professionnalisation de ceux-ci qui est en voie. Cependant nous nous
sommes heurté aux regles bien défini du projet de communication qui
devait le plus possible ce déroulé loin de l'informatique et du monde
numérique, autant dire que nous ne partions pas gagnant même si nous
étions très motivé et déterminé à convaincre nos enseignants. Il y a eu
des moments d'incertitudes mais nous nous sommes efforcé de nous motiver
les uns les autres. Nous avons du défendre à plusieurs reprises nos
idées comparées aux autres projet ce qui nous à demandé d'être clair sur
nos intentions.


% subsection la_soumission_de_leidee (end)%}}}
\subsection{Projet accepté}%{{{
\label{sub:projet_accepte}

Notre projet à mis plus de temps que les autres a être validé.
Celui-ci devait être soumis à deux personnes. Nous nous sommes retrouvé
pendant un laps de temps (2 semaines) sans savoir si notre projet était
validé ou non. Nous étions inquiêt a son sujet parceque si il n'était
pas accepté, cela nous aurait mis grandement en retard pour tout dabord
en trouver un nouveau mais aussi le mettre sur pied. Nous avions pensé à
plusieurs solution de remplacement mais celle-ci nous motivait pas
autant voir pas du tout comparé à notre idée principale.

Nous avions pensé à :
  - tournois de tennis de table : Deux membre du groupe en ont fait
en club et aime ce sport cependant il est très difficile de se procuré
du materiel pour tout un tournoi.
  - Post-it war : le concept est de proposer au gens des différents
batiments de lille1 de déssiner avec des post-it de couleur. Les dessins
aurait été soumis à des juges qui aurait décerné la meilleur oeuvre.
Pour l'anecdote, cette idée à été réalité à la fin de cette année par
l'association AEI (lien vers description)

Heureusement pour nous, nous n'avons pas eu besoin d'y reflechir
d'avantage et nous avons du nous mêttre au travail pour réaliser notre
projet.

% subsection projet_accepte (end)%}}}
