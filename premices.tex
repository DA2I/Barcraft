\section{L'objectif du projet de communication}%{{{
\label{sec:l_objectif_du_projet_de_communication}

Les principaux objectifs du projet de communication sont :

\begin{enumerate}

\item Promouvoir la formation au travers d'activités tantôt éducatives,
tantôt ludique ;

\item S'ouvrir au monde extérieur en montrant que des projets en dehors
de l'informatique sont menées au sein de la licence professionnelle ;

\item Faire face à des responsabilités et à la pression que celle-ci
peut entrainer ;

\item Mener de bout en bout l'élaboration d'un projet, qui plus est en
groupe.

\end{enumerate}

% section l_objectif_du_projet_de_communication (end)%}}}
\section{L'idée principale}
\label{sec:l_idee_principale}

\lettrine{A}{yant} une passion commune autour de l'eSport\, \footnote{Ce
mot est expliqué en section \ref{sec:le_sport_electronique_esport_} à la
page \pageref{sec:le_sport_electronique_esport_}.} et des jeux vidéos en
général, nous cherchions un projet ambitieux et original pour promouvoir
la formation.  Nous ne voulions pas d'un projet maintes et maintes fois
réalisé et finalement peu motivant qui aurait reflété en nous
l'obligation de faire plutôt que l'envie de faire.  Ayant suivi avec
attention la montée en puissance d'un nouveau phénomène qu'est le
Barcraft\, \footnote{Expliqué dans la même partie à la page
\pageref{sec:barcraft}.} nous nous sommes mis en tête de présenter cette
idée et de la défendre.

% section l_idee_principale (end)
\section{Les jeux considérés}%{{{
\label{sec:les_jeux_consideres}

\lettrine{L}{ors} de nos réunions nous avons eu plusieurs idées de jeux
à présenter pendant la soirée. En effet, nous nous sommes dit que
Starcraft (nous l'expliquons un peu plus loin) était peut-être un
segment trop étroit pour attirer du monde et qu'il fallait mieux
présenter plusieurs jeux et proposer des plages horaires pour ceux-ci.
Nous allons voir plus loin que cette idée ne fut pas retenue au final.

\begin{description}

\item[Starcraft - Brood War :] Celui-ci devait être présenté pour
montrer les différences de graphismes et de gameplay de la franchise au
fil des années mais il n'y avait qu'une personne du groupe qui possédait
le jeu -- qui est en plus décomposé en un jeu plus une extension -- ce
qui fait que les autres devait l'acquérir pour la soirée.  De plus, \og
Brood War \fg{} par son graphisme 2D aurait surement fait fuir de
panique le public. Nous en parlons un peu plus dans la section
\emph{Présentation}.

\item[Starcraft II :] Le jeu principal auquel tout les membres du groupe
joue et connaissent bien. Celui-ci est décomposé en deux jeux distinct.
La partie la plus intéressante étant le jeu en ligne bien évidement.
L'extension \og Heart of the Swarm \fg{} étant sorti la veille de la
soirée il a fallu nous empresser de l'acheter pour pouvoir y jouer et
ainsi montrer les nouvelles unités du jeu. Ce jeux est le premier auquel
nous ayons pensé tout simplement parce que c'est celui-ci qui à
démocratisé le concept de Barcraft.

\item[League of Legends :] C'est un jeu très connu qui aurait attiré
beaucoup de monde car plus accessible aux premiers abords. Seulement, il
n'y a qu'une personne du groupe qui y joue. Il fallait aussi trouver des
joueurs et un commentateur spécifiquement pour celui-ci. \emph{LoL} est en plus
un jeu qui se joue en équipe de cinq ce qui complexifie grandement la
recherche d'intervenants.

\item[Retro-gaming :] On voulait proposer d'autre jeux de stratégies
d'anthologies à côté des matchs commentés mais ça aurait dénaturer le
projet pour finir en soirée jeux vidéos ce qui n'a jamais été notre
intention. De plus, il aurait fallu encadrer un nombre conséquent de
matériel durant la soirée ce qui aurait été dur à gérer à trois
personnes.

\item[Jeu de plateau :] Nous avons pensé à nous procurer des jeux de
plateaux. En effet, l'association \emph{dès à la carte} organise ce
genre d'événement assez régulièrement. Nous avons décidé assez vite de
ne pas retenir cette idée car cela aurait causé trop de dispersion dans
la soirée et le concept de Barcraft n'aurait plus de sens.

\end{description}

Au final nous sommes revenu en arrières sur nos idées. Nous avons décidé
de ne présenter que Starcraft II qui est le jeux que nous connaissons le
plus. Cependant certains des jeux ont été encore proposés un mois avant
la date buttoir, c'est pourquoi ceux-ci sont présenté dans ce rapport.

% section les_jeux_consideres (end)%}}}
\section{La soumission de l'idée}%{{{
\label{sec:la_soumission_de_l_idee}

\lettrine{N}{ous} avons cherché à démystifier la pratique du jeux vidéos
qui est soumis malgré elle à un certain nombre d'idées reçues et de
préjugées.  Il fallait mettre en avant l'aspect stratégique des jeux
choisi et la professionnalisation de ceux-ci qui est en voie. Cependant
nous nous sommes heurté aux règles bien défini du projet de
communication qui devait le plus possible ce déroulé loin de
l'informatique et du monde numérique, autant dire que nous ne partions
pas gagnant même si nous étions très motivé et déterminé à convaincre
nos enseignants. Il y a eu des moments d'incertitudes mais nous nous
sommes efforcé de nous motiver les uns les autres. Nous avons du
défendre à plusieurs reprises nos idées comparées aux autres projet ce
qui nous à demandé d'être clair sur nos intentions.

% section la_soumission_de_l_idee (end)%}}}
\section{Projet accepté}%{{{
\label{sec:projet_accepte}

\lettrine{N}{otre} projet a mis plus de temps que les autres a être
validé.  Celui-ci devait être soumis à deux personnes. Nous nous sommes
retrouvé pendant un laps de temps (2 semaines) sans savoir si notre
projet était validé ou non. Nous étions inquiet parce que si il n'était
pas accepté, cela nous aurait mis grandement en retard pour tout d'abord
en trouver un nouveau mais aussi mettre ce nouveau projet sur pied. Nous
avions pensé à plusieurs solution de remplacement mais celle-ci nous
motivait pas autant voir pas du tout comparé à notre idée principale.

Nous avions pensé à :

\begin{description}

\item[Tournois de tennis de table :] Deux membre du groupe en ont fait en
club et aime ce sport cependant il est très difficile de se procurer du
matériel pour tout un tournoi ;

\item[Post-it war :] le concept est de proposer au gens des différents
bâtiments de lille 1 de faire de grands dessins sur les baies vitrées
avec des post-it de couleurs. Les dessins aurait été soumis à des juges
qui aurait décerné la meilleur \oe{}uvre.  Pour l'anecdote, cette idée à
été réalité à la fin de cette année par l'AEI.

\end{description}

Heureusement pour nous, nous n'avons pas eu besoin d'y réfléchir
d'avantage et nous avons du nous mettre au travail pour réaliser notre
projet.

% section projet_accepte (end)%}}}
