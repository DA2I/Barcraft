\section{L'objectif du projet de communication}%{{{
\label{sec:l_objectif_du_projet_de_communication}

Les principaux objectifs du projet de communication sont :

\begin{enumerate}

\item Promouvoir la formation au travers d'activités tantôt éducatives
tantôt lucrative ;

\item S'ouvrir au monde extérieur en montrant que des projets en dehors
de l'informatique sont menées au sein de la licence professionnelle ;

\item Faire face a des responsabilités et à la pression que celle-ci
peut entrainer ;

\item Mener de bout en bout l'élaboration d'un projet, qui plus est en
groupe.

\end{enumerate}

% section l_objectif_du_projet_de_communication (end)%}}}
\section{L'idée principale}
\label{sec:l_idee_principale}

\lettrine{A}{yant} une passion commune autour de l'eSport\, \footnote{Ce
mot est expliqué en section \ref{sec:le_sport_electronique_esport_} à la
page \pageref{sec:le_sport_electronique_esport_}.} et des jeux vidéos en
général, nous cherchions un projet ambitieux et original pour promouvoir
la formation. Nous ne voulions pas d'un projet maintes et maintes fois
réalisé et finalement peu motivant qui aurait reflété en nous
l'obligation de faire plutôt que l'envie de faire. Ayant suivi avec
attention la montée en puissance d'un nouveau phénomène qu'est le
Barcraft\, \footnote{Expliqué dans la même partie à la page
\pageref{sec:barcraft}.} nous nous sommes mis en tête de présenter cette
idée et de la défendre.

% section l_idee_principale (end)
\section{Les jeux considérés}%{{{
\label{sec:les_jeux_consideres}

\lettrine{L}{ors} de nos réunions nous avons eu plusieurs idées de jeux
à présenter pendant la soirée. En effet, nous nous sommes dit que
Starcraft (nous expliquons un peu plus loin ce que c'est) était
peut-être un segment trop étroit pour attirer du monde et qu'il valait
mieux présenter plusieurs jeux en proposant des plages horaires pour
ceux-ci. Nous allons voir plus loin que cette idée ne fut pas retenue au
final.

\begin{description}

\item[Starcraft - Brood War :] Celui-ci devait être présenté pour
montrer les différences de graphismes et de gameplay de la franchise au
fil des années mais il n'y avait qu'une personne du groupe qui possédait
le jeu -- qui est en plus décomposé en un jeu + une extension -- ce qui
fait que les autres devaient l'acquérir pour la soirée. De plus, \og
Brood War \fg{} par son graphisme 2D aurait surement fait fuir le public. 
Celui-ci est présenté dans la section
\emph{Présentation}.

\item[Starcraft II :] Le jeu principal auquel tous les membres du groupe
jouent et connaissent bien. Celui-ci est décomposé en deux jeux distincts.
La partie la plus intéressante étant le jeu en ligne bien évidemment.
L'extension \og Heart of the Swarm \fg{} étant sortie la veille de la
soirée il a fallu nous empresser de l'acheter pour pouvoir y jouer et
ainsi montrer les nouvelles unités du jeu. Ce jeux est le premier auquel
nous ayons pensé tout simplement parce que c'est celui-ci qui a
démocratisé le concept de Barcraft.

\item[League of Legends :] C'est un jeu très connu qui aurait attiré
beaucoup de monde car plus accessible au premier abord. Seulement, il
n'y a qu'une personne du groupe qui y joue. Il fallait aussi trouver des
joueurs et un commentateurs spécifiquement pour celui-ci. \emph{LoL} est en plus
un jeu qui se joue en équipe de cinq ce qui complexifie grandement la
recherche d'intervenants.

\item[Retro-gaming :] On voulait proposer d'autres jeux de stratégie
d'anthologie à côté des matchs commentés mais cela aurait dénaturé le
projet pour finir en soirée jeux vidéo, ce qui n'a jamais été notre
intention. De plus, il aurait fallu encadrer un nombre conséquent de
matériel durant la soirée ce qui aurait été dur à gérer à trois
personnes.

\item[Jeu de plateau :] Nous avons pensé à nous procurer des jeux de
plateaux. En effet, l'association \emph{Dés à la Carte} organise ce
genre d'événement assez régulièrement. Nous avons décidé assez vite de
ne pas retenir cette idée car cela aurait causé trop de dispersions dans
la soirée et le concept de Barcraft n'aurait plus eu de sens.

\end{description}

Au final nous sommes revenus sur nos idées. Nous avons décidé
de ne présenter que Starcraft II qui est le jeu que nous connaissons le
plus. Cependant certains des jeux ont été encore proposés un mois avant
la date butoire, c'est pourquoi ceux-ci sont présentés dans ce rapport.

% section les_jeux_consideres (end)%}}}
\section{La soumission de l'idée}%{{{
\label{sec:la_soumission_de_l_idee}

\lettrine{N}{ous} avons cherché à démystifier la pratique des jeux vidéo, 
soumise malgré elle à un certain nombre d'idées reçues et de
préjugés. Il fallait mettre en avant l'aspect stratégique des jeux
choisis et la professionnalisation de ceux-ci qui est en pleine expansion. 
Cependant nous nous sommes heurtés aux règles bien définies du projet de
communication qui devait le plus possible se dérouler loin de
l'informatique et du monde numérique, autant dire que nous ne partions
pas gagnants même si nous étions très motivés et déterminés à convaincre
nos enseignants. Il y a eu des moments d'incertitudes mais nous nous
sommes efforcés à nous motiver les uns les autres. Nous avons dû
défendre à plusieurs reprises nos idées comparées aux autres projets ce
qui nous a demandé une plus grande clarté sur nos intentions.

% section la_soumission_de_l_idee (end)%}}}
\section{Projet accepté}%{{{
\label{sec:projet_accepte}

\lettrine{N}{otre} projet a mis plus de temps que les autres à être
validé. Celui-ci devait être soumis à deux personnes. Nous nous sommes
retrouvé pendant un laps de temps (2 semaines) sans savoir si notre
projet était validé ou non. Nous étions inquiets parce que s'il n'était
pas accepté, cela nous aurait mis grandement en retard pour 
en trouver un nouveau mais également mettre ce nouveau projet sur pied. Nous
avions pensé à plusieurs solutions de remplacement mais celles-ci ne nous
motivait pas autant voir pas du tout comparé à notre idée principale.

Nous avions pensé à :

\begin{description}

\item[Tournois de tennis de table :] Deux membres du groupe en ont fait en
club et aimennt ce sport cependant il est très difficile de se procurer du
matériel pour tout un tournoi ;

\item[Post-it war :] le concept est de proposer aux gens des différents
bâtiments de Lille 1 de faire de grands dessins sur les baies vitrées
avec des post-it de couleurs. Les dessins aurait été soumis à des juges
qui aurait décerné la meilleur \oe{}uvre.  Pour l'anecdote, cette idée à
été réalité à la fin de cette année par l'AEI.

\end{description}

Heureusement pour nous, nous n'avons pas eu besoin d'y réfléchir
d'avantage et nous avons du nous mettre au travail pour réaliser notre
projet.

% section projet_accepte (end)%}}}
