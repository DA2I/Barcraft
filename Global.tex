\documentclass[12pt,a4paper]{report}
\usepackage[utf8]{inputenc}
\usepackage[frenchb]{babel}
%\usepackage{baskervald}
\usepackage{gfsartemisia}
%\usepackage{palatino}
%\usepackage{libertine}
\usepackage[T1]{fontenc}
\usepackage{textcomp} % Pour les symboles spéciaux
\usepackage{listings} % Pour l'ajout de code source
\usepackage{graphicx} % Pour l'ajout des images
\usepackage{titlesec} % Pour ne plus avoir le mot chapitre
\usepackage{lettrine} % Pour l'usage des lettrines

% Pas de mot 'chapitre'
\titleformat{\chapter}[hang]{\bf\huge}{\thechapter}{2pc}{}

% Marge pour la reliure
\addtolength{\hoffset}{+0.5cm}

% Style de page personalisé%{{{
\usepackage{fancyhdr}
\pagestyle{fancy} % Ceci permet d’avoir les noms de chapitre et de section % en minuscules
\renewcommand{\chaptermark}[1]{\markboth{#1}{}}
\renewcommand{\sectionmark}[1]{\markright{\thesection\ #1}}
\fancyhf{}	% supprime les en-têtes et pieds
\fancyhead[LE,RO]{\bfseries\thepage}% Left Even, Right Odd
\fancyhead[LO]{\bfseries\rightmark} % Left Odd
\fancyhead[RE]{\bfseries\leftmark} % Right Even
\renewcommand{\headrulewidth}{0.5pt}% filet en haut de page
\addtolength{\headheight}{0.5pt}	% espace pour le filet
\renewcommand{\footrulewidth}{0pt} % pas de filet en bas
\fancypagestyle{plain}{ % pages de tetes de chapitre
\fancyhead{}	% supprime l’entete
\renewcommand{\headrulewidth}{0pt} % et le filet
}%}}}

%Couverture%{{{

\title
{
	\normalsize{IUT A Lille1 - Villneuve-d'Ascq\\
	2012-2013}\\
	\vspace{15mm}
  \Huge{Barcraft
    \vspace{15mm}}
}
\author{
\bsc{Stechele} Julien\\
\bsc{Vanuxem} Yannick\\
\bsc{Serir} Jean-francois\\
	\vspace{30mm}
}
%}}}

\begin{document}%{{{

\maketitle

%\input{remerciement}

% Pour écrire Sommaire au lieu de table des matières
\renewcommand{\contentsname}{Sommaire}

% Interligne pour le sommaire
%{\setlength{\baselineskip}{1.9\baselineskip}\tableofcontents\par}

\section{Les prémices du projet}%{{{
\label{sec:les_premices_du_projet}

  \section{L'objectif du projet de communication}%{{{
\label{sec:l_objectif_du_projet_de_communication}

Les principaux objectifs du projet de communication sont :

\begin{enumerate}

\item Promouvoir la formation au travers d'activités tantôt éducatives
tantôt lucrative ;

\item S'ouvrir au monde extérieur en montrant que des projets en dehors
de l'informatique sont menées au sein de la licence professionnelle ;

\item Faire face a des responsabilités et à la pression que celle-ci
peut entrainer ;

\item Mener de bout en bout l'élaboration d'un projet, qui plus est en
groupe.

\end{enumerate}

% section l_objectif_du_projet_de_communication (end)%}}}
\section{L'idée principale}
\label{sec:l_idee_principale}

\lettrine{A}{yant} une passion commune autour de l'eSport\, \footnote{Ce
mot est expliqué en section \ref{sec:le_sport_electronique_esport_} à la
page \pageref{sec:le_sport_electronique_esport_}.} et des jeux vidéos en
général, nous cherchions un projet ambitieux et original pour promouvoir
la formation. Nous ne voulions pas d'un projet maintes et maintes fois
réalisé et finalement peu motivant qui aurait reflété en nous
l'obligation de faire plutôt que l'envie de faire. Ayant suivi avec
attention la montée en puissance d'un nouveau phénomène qu'est le
Barcraft\, \footnote{Expliqué dans la même partie à la page
\pageref{sec:barcraft}.} nous nous sommes mis en tête de présenter cette
idée et de la défendre.

% section l_idee_principale (end)
\section{Les jeux considérés}%{{{
\label{sec:les_jeux_consideres}

\lettrine{L}{ors} de nos réunions nous avons eu plusieurs idées de jeux
à présenter pendant la soirée. En effet, nous nous sommes dit que
Starcraft (nous expliquons un peu plus loin ce que c'est) était
peut-être un segment trop étroit pour attirer du monde et qu'il valait
mieux présenter plusieurs jeux en proposant des plages horaires pour
ceux-ci. Nous allons voir plus loin que cette idée ne fut pas retenue au
final.

\begin{description}

\item[Starcraft - Brood War :] Celui-ci devait être présenté pour
montrer les différences de graphismes et de gameplay de la franchise au
fil des années mais il n'y avait qu'une personne du groupe qui possédait
le jeu -- qui est en plus décomposé en un jeu + une extension -- ce qui
fait que les autres devaient l'acquérir pour la soirée. De plus, \og
Brood War \fg{} par son graphisme 2D aurait surement fait fuir le public. 
Celui-ci est présenté dans la section
\emph{Présentation}.

\item[Starcraft II :] Le jeu principal auquel tous les membres du groupe
jouent et connaissent bien. Celui-ci est décomposé en deux jeux distincts.
La partie la plus intéressante étant le jeu en ligne bien évidemment.
L'extension \og Heart of the Swarm \fg{} étant sortie la veille de la
soirée il a fallu nous empresser de l'acheter pour pouvoir y jouer et
ainsi montrer les nouvelles unités du jeu. Ce jeux est le premier auquel
nous ayons pensé tout simplement parce que c'est celui-ci qui a
démocratisé le concept de Barcraft.

\item[League of Legends :] C'est un jeu très connu qui aurait attiré
beaucoup de monde car plus accessible au premier abord. Seulement, il
n'y a qu'une personne du groupe qui y joue. Il fallait aussi trouver des
joueurs et un commentateurs spécifiquement pour celui-ci. \emph{LoL} est en plus
un jeu qui se joue en équipe de cinq ce qui complexifie grandement la
recherche d'intervenants.

\item[Retro-gaming :] On voulait proposer d'autres jeux de stratégie
d'anthologie à côté des matchs commentés mais cela aurait dénaturé le
projet pour finir en soirée jeux vidéo, ce qui n'a jamais été notre
intention. De plus, il aurait fallu encadrer un nombre conséquent de
matériel durant la soirée ce qui aurait été dur à gérer à trois
personnes.

\item[Jeu de plateau :] Nous avons pensé à nous procurer des jeux de
plateaux. En effet, l'association \emph{Dés à la Carte} organise ce
genre d'événement assez régulièrement. Nous avons décidé assez vite de
ne pas retenir cette idée car cela aurait causé trop de dispersions dans
la soirée et le concept de Barcraft n'aurait plus eu de sens.

\end{description}

Au final nous sommes revenus sur nos idées. Nous avons décidé
de ne présenter que Starcraft II qui est le jeu que nous connaissons le
plus. Cependant certains des jeux ont été encore proposés un mois avant
la date butoire, c'est pourquoi ceux-ci sont présentés dans ce rapport.

% section les_jeux_consideres (end)%}}}
\section{La soumission de l'idée}%{{{
\label{sec:la_soumission_de_l_idee}

\lettrine{N}{ous} avons cherché à démystifier la pratique des jeux vidéo, 
soumise malgré elle à un certain nombre d'idées reçues et de
préjugés. Il fallait mettre en avant l'aspect stratégique des jeux
choisis et la professionnalisation de ceux-ci qui est en pleine expansion. 
Cependant nous nous sommes heurtés aux règles bien définies du projet de
communication qui devait le plus possible se dérouler loin de
l'informatique et du monde numérique, autant dire que nous ne partions
pas gagnants même si nous étions très motivés et déterminés à convaincre
nos enseignants. Il y a eu des moments d'incertitudes mais nous nous
sommes efforcés à nous motiver les uns les autres. Nous avons dû
défendre à plusieurs reprises nos idées comparées aux autres projets ce
qui nous a demandé une plus grande clarté sur nos intentions.

% section la_soumission_de_l_idee (end)%}}}
\section{Projet accepté}%{{{
\label{sec:projet_accepte}

\lettrine{N}{otre} projet a mis plus de temps que les autres à être
validé. Celui-ci devait être soumis à deux personnes. Nous nous sommes
retrouvé pendant un laps de temps (2 semaines) sans savoir si notre
projet était validé ou non. Nous étions inquiets parce que s'il n'était
pas accepté, cela nous aurait mis grandement en retard pour 
en trouver un nouveau mais également mettre ce nouveau projet sur pied. Nous
avions pensé à plusieurs solutions de remplacement mais celles-ci ne nous
motivait pas autant voir pas du tout comparé à notre idée principale.

Nous avions pensé à :

\begin{description}

\item[Tournois de tennis de table :] Deux membres du groupe en ont fait en
club et aimennt ce sport cependant il est très difficile de se procurer du
matériel pour tout un tournoi ;

\item[Post-it war :] le concept est de proposer aux gens des différents
bâtiments de Lille 1 de faire de grands dessins sur les baies vitrées
avec des post-it de couleurs. Les dessins aurait été soumis à des juges
qui aurait décerné la meilleur \oe{}uvre.  Pour l'anecdote, cette idée à
été réalité à la fin de cette année par l'AEI.

\end{description}

Heureusement pour nous, nous n'avons pas eu besoin d'y réfléchir
d'avantage et nous avons du nous mettre au travail pour réaliser notre
projet.

% section projet_accepte (end)%}}}


% section les_premices_du_projet (end)%}}}

\section{Présentation}%{{{
\label{sec:presentation}

  \subsection{Jeux de stratégie en temps réel}%{{{
\label{sub:jeux_de_strategie_en_temps_reel}

Une très bonne définition de ce qu'est le jeux de stratégie en
temps réel est consultable sur wikipédia :

  Le jeu de stratégie en temps réel (STR, ou RTS pour la
  dénomination du genre en anglais : real-time strategy) est un type de
  jeu de stratégie particulier qui notamment et par opposition au jeu de
  stratégie au tour par tour n’utilise pas un découpage arbitraire du
  temps [...]

  Dans le tour par tour, donc, l’issue d’une confrontation est résolue par
  étapes, un combat en succédant un autre, de manière à laisser à chaque
  joueur le temps de réfléchir sur la prochaine étape. Une confrontation
  est alors résolue par combats successifs et isolés.

  Mais le problème qui se pose lorsqu’on veut simuler des affrontements
  réalistes est qu’en général, les unités n’attendent pas que ce soit leur
  tour pour attaquer et gérer l’affrontement de plusieurs unités en même
  temps et de façon réaliste relève du domaine de l’impossible pour l’être
  humain, jusqu’à l’invention des processeurs. Ceux-ci, programmés de
  manière à simuler un affrontement peuvent désormais gérer le déplacement
  de plusieurs unités simultanément et résoudre les combats de façon
  simultanée et rigoureuse dans le cadre d’un jeu vidéo.

% subsection jeux_de_strategie_en_temps_reel (end)%}}}
\subsection{La licence Starcraft}%{{{
\label{sub:la_licence_starcraft}

Starcraft et Starcraft: Brood War

  Starcraft est un jeu vidéo de stratégie en temps réel (STR) développé
  par Blizzard Entertainment. Le premier opus est sorti en Europe le 31
  mars 1998 sur PC. Il s'inscrit dans la lignée des jeux de strategie de
  Blizzard maintenant très populaire auprès des amateurs du genre. Avec
  plus de 11 millions de copies vendues dans le monde, il est un des
  jeux vidéo sur PC les mieux vendus et reste à ce jour le jeu de
  stratégie en temps réel le plus vendu de tout les temps.

  (image)

Starcraft 2 : (Wings of Liberty)

  Celui-ci sort dans le monde après 12 ans d'attente
  interminable le 27 juillet 2010. Blizzard ayant fait le choix marketing
  de faire trois campagnes distinctes, il y aura deux extensions de
  prevues ce qui à fait polémique à l'époque. Avec le deuxième opus,
  blizzard à fait le pari de démocratisé le jeux de stratégie en temps
  réel en rendant WoL plus facile d'accès comparées a Brood War qui lui
  requiérait beaucoup d'APM (nbp). Pari reussi puisqu'il connait
  rapidement un important succès commercial avec plus de 4,5 millions de
  copies vendues 6 mois après la sortie du jeu.

  (image)

Starcraft 2 : (Heart of the Swarm)

  Heart of the Swarm est la première des deux extensions prévu
  du jeux Starcraft 2. Elle constitue la suite du premier chapitre Wings
  of Liberty. La date de sortie officielle est le 12 mars 2013. Cette opus
  apporte une nouvelle campagne et des nouvelles unités au mode
  multijoueur. Aucune statistique de vente n'a encore été publiée par
  blizzard actuellement.


% subsection la_licence_starcraft (end)%}}}
\subsection{League of Legends}%{{{
\label{sub:league_of_legends}

- présentation rapide du jeu

% subsection league_of_legends (end)%}}}
\subsection{Barcraft}%{{{
\label{sub:barcraft}

Comme sous nom l'indique, il s'agit d'organiser une soirée autour
du jeu starcraft avec l'ambiance d'un bar.	Pour cela, nous faisons
appelle à des intervenants extérieurs. Afin d'ovrir un maximum de
spectacle, nous avons demander à des joueurs professionnels ainsi
que des commentateurs professionnels de participer à l'evénement.
La soirée se déroule sous la forme de matchs commentés. Pour une
bonne intéraction avec le public, les matchs sont vidéoprojetés sur
un écran géant. Au cour de la soirée les joueurs du public pouront
prendre place et jouer contre notre joueur professionnel.

% subsection barcraft (end)%}}}


% section presentation (end)%}}}

\section{Organisation}%{{{
\label{sec:organisation}

  \subsection{Technique}%{{{
\label{sub:technique}

Technique
  - Connexion internet avec certains ports ouverts
  - Premier test avec une adresse public en salle de tp concluant
  - Pour le deploiement vers la mde, le responsable du réseau ne voulait pas nous fournir 4 ip public
  - Mise au point d'une autre solution
         - VLAN + NAT
  - Test en salle tp concluant
  - Brassage des prises de la mde sur le VLAN
  - Test à la mde concluant

% subsection technique (end)%}}}
\subsection{Materiel}%{{{
\label{sub:materiel}

- ordinateurs puissants
- videoprojecteur
- ampli + micro
- tables et chaises
- connectique
- isolation des joueurs

% subsection materiel (end)%}}}
\subsection{Salle et bar} %{{{
\label{sub:salle_et_bar}

Nous cherchions un local qui aurait le maximum de materiel dont nous
avions besoin pour la soirée pour plus de facilité. Nous avons tout
dabord cherché a l'interieur du campus de lille avant de vouloir trouvé
un commerce qui nous aurait demandé beaucoup plus d'éffort d'un point de
vue recherche.

Ayant participé a certaines soirées étudiantes qui se déroule le jeudi
soir a la maison des étudiants (MDE), nous avons remarqué que celle-ci
était disponible assez facilement, gratuitement et avec du materiel
approprié. En effet, les thématiques proposées par la MDE sont très
diverses. Elle peut aussi bien faire office de salle de reunion que de
concert ou même de soirée cinéma.

Cependant, pour avoir le droit d'organiser quelquechose dans celle-ci il
faut appartenir à une association étudiante connu du campus. Aucun
membre du groupe étant licencé dans ce genre d'initiative nous étions un
peu inquiet. Après s'être renseigné a differents endroits, il faut
simplement se rapprocher d'une association pour qu'elle puisse se porte
garant de l'évenement.

La nécéssité d'avoir une association derrière nous est obligatoire pour
l'obtention d'une extension d'assurance pour les biens et les personnes
au cas ou un problème majeur surviendrait lors de la soirée.

- redaction de la convention avec l'aei
- L'AMUL
- Plusieurs mails et rendez-vous

% subsection salle_et_bar (end)%}}}
\subsection{Partenariat et sponsor} %{{{
\label{sub:partenariat_et_sponsor}

  % TODO
- Demande de financement du projet auprès de *JE SAIS PLUS QUOI*

  Nous nous sommes tout dabord rendu au batiment A3
  pour récupéré un dossier de demande de financement pour les projets des
  étudiants. Ce dossier requierait un certain nombre de pièces
  justificative comme le planning de déroulement de la soirée, le montant
  envisagé ainsi que le potentiel nombre de participant. Par la suite un
  entretiens avec les membres du groupe face à un jury de *JE SAIS
  PLUS QUOI*. Nous avons abandonnée cette possibilité car nous ne savions
  pas si le projet était possible (connexion).

  %TODO
- collaboration avec l'aei

  Nous avons choisi de traiter avec l'AEI (association des étudiants
  en informatique) qui sont des habituées de ce genre de soirée à
  thématique *geek* et emprunteur historique de la maison des étudiants.
  Leur principal local ce situe dans le batiment M5 au milieu de la
  cité scientifique ou sont concentré la pluspart des informaticiens de
  lille1.
  Nous avons donc 

  %TODO
- demande de financement auprès de Micromania (v2)

  Une demande de financement à été faite auprès du célèbre magasin de
  jeux-vidéo Micromania qui nous a vallu un refus cathégorique. Cette
  enseigne ne sponsorisant pas d'évenement par soucis
  budgétaire. (j'invente)

  %TODO
- demande de sponsor Materiel.net (lomme)

  %TODO
- envoie d'un mail au service communication sans reponse

  N'ayant trouvé aucun moyen de finacement pour permettre aux
  intervenants et nous même de nous restaurer lors de l'évenement, nous
  n'avions pas d'autre choix que de répartir les coûts entre membre du
  projet. C'est à ce moment la que le réponsable de la formation DA2I
  proposa de débloqué les fonds allouées à la licence pour permettre de
  financer les évnements des étudiants. Ce fut une aubaine pour nous.

  Nous avons tout dabord pensé à acheter nous même les aliments pour
  ensuite ce faire rembourser avec cette somme mais cela posait des
  problèmes logistiques. Une autre idée à donc été proposé : commander des
  plateaux repas (qui est un service proposé par l'iut). Il nous a donc
  été demandé de réunir des informations auprès des intervenants de la
  soirée, comme leur nom, prenom, adresse, fonction pour que cette demande
  aboutisse.

% subsection partenariat_et_sponsor (end)%}}}
\subsection{Intervenants}%{{{
\label{sub:intervenants}

Leur role dans la soirée

  Nous souhaitions invité des personnes exterieures pour le projet à
  differente fins. Il nous fallait un ou des commentateurs pour les
  parties ainsi que des joueurs. Ceux-ci, pour engrangé du monde, devait
  être connu de la scène française.


- Alexandre "Makoz" Chilling
  - Voir les échanges facebook

Un des membres du groupe qui, connaissant bien la scène française
autour de starcraft 2, s'est occupé de contacter des intervenants. Le
role

- team *aAa* demande 500€
                - pas de négociation possible
                - somme > budget prévisionnel
                - refus de l'offre
- Thomas Brandt aka soey & Julien roy (lieu dunkerque) (demande 0€)
    - Acceptation de l'offre


% subsection intervenants (end)%}}}
\subsection{Communication}%{{{
\label{sub:communication}

Communication
        - Mail general a tous les etudiants (newslettre mde)
        - affiche A3, imprimé à l'IUT (30 unités disposées dans la fac)
        - questionnaire diffusé à tous les étudiants
        - Facebook : création d'un evenement (MORT)

% subsection communication (end)%}}}
\subsection{Bilan financier}%{{{
\label{sub:bilan_financier}

  %TODO
Bilan financier
        Prévisionnel
            - salle
            - bar
            - matos
            - intervenant
            - comm
            - repas
        Reel
           ...

% subsection bilan_financier (end)%}}}
\subsection{Déroulement de la soirée}%{{{
\label{sub:deroulement_de_la_soiree}

        Previsionnel
            18h/22h
            arrivé 16h
            depart 23h
            deroulement
                - videos d'introduction au jeu pour les neophytes
                - parties commentées avec des joueurs professionnels
                - parties commentées avec des joueurs du public

        Imprevu
           Problème de connexion battle.net
           Meteo defavorable
           Intervenants pas présent
           impac sur le déroulement prévisionnel

        Réel
            18h/23h20
            arrivé 16h15
            départ 23h55
            deroulement
                - videos d'untroduction au jeu pour les neophytes
                - parties commentées par julien avec le public
                - coupure pour manger vers 21h pendant 20 minutes

% subsection deroulement_de_la_soiree (end)%}}}


% section organisation (end)%}}}

\section{Bilan}%{{{
\label{sec:bilan}

  \input{Bilan}

% section bilan (end)%}}}

% fake annexe
\appendix

% Conseils pour réussir ce projet

\end{document}%}}}
